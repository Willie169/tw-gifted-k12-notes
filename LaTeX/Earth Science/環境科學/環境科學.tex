\documentclass[a4paper,12pt]{report}
\setcounter{secnumdepth}{5}
\setcounter{tocdepth}{3}
\newcounter{ZhRenew}
\setcounter{ZhRenew}{1}
\newcounter{SectionLanguage}
\setcounter{SectionLanguage}{1}
\input{/usr/share/latex-toolkit/template.tex}
\begin{document}
\title{環境科學}
\author{沈威宇}
\date{\temtoday}
\titletocdoc
\chapter{環境科學(Environmental Science)}


\section{自然資源(Natural resources)}
\subsection{資源分類}
\begin{itemize}
\item \tb{再生(Renewable)資源}:短期間內可再生或循環使用。如土地、生物、水、空氣與太陽能資源。
\item \tb{非再生資源}:短期間內不可再生或循環使用。如礦產資源。
\item \tb{能源}:提供能量的資源。
\item \tb{再生能源/替代能源}:再生資源與能源的交集。如太陽能、生質能、地熱能、海洋能、風力能和水力能。
\item \tb{非再生能源}:非再生資源與能源的交集。如石油、天然氣、煤等化石能源及鈾礦等。國際能源署(IEA)預估2035年化石能源使用比例仍占總能源七成以上。
\item \tb{初級能源}:在自然界天然形成的能源。如石油、天然氣、煤、太陽能。
\item \tb{次級能源}:初級能源經過人類加工處理而得的能源。如電能、汽油。
\item \tb{合理開發}:技術上可行,不會浪費資源,不會造成無法復原的環境破壞,兼容世代正義。
\end{itemize}
\subsection{資源回收}
\begin{itemize}
\item \tb{電子廢棄物}:每年全球約五百億公斤,回收率僅約百分之十六,德國2010電子垃圾回收率約百分之四十五。
\item \tb{加熱或冷卻用水}:通常不含太多汙染物,通常可直接用於灌溉、清洗等。
\item \tb{有機物}:可引入一系列滯留時間不同的水池,分別讓嗜氧菌和厭氧菌對水中汙染物進行分解,產生甲烷、二氧化碳、硫化氫、乙酸、氫氣、水等,沼氣可作為燃料,水體可用於灌溉、工業、清洗等。
\end{itemize}
\subsection{能源}
\subsubsection{核能發電}
使用核分裂的熱加熱水產生蒸氣推動渦輪機發電。成熟有效、能量密度極高,且不排放二氧化碳,初置成本較火力發電高,有游離輻射外洩疑慮,尤其地震帶,並有放射性核廢料儲存問題。
\sssc{核融合發電}
與核分裂相比,無放射性廢料,現尚未能控制反應,仍在實驗階段。
\subsubsection{化石燃料}
燃燒化石燃料,如燃煤、燃氣、燃油,可推動渦輪機發電或交通運輸等用途。
\subsubsection{氫能}
燃燒氫氣之熱值140kJ/g,遠高於化石燃料,反應主要產物為水,汙染低。生產氫氣技術多元,如水的電解、化石燃料脫氫、光觸媒分解水等,也是許多工業製程的副產物。但氫氣分子小而易燃,儲氫設備和後端應用需配合氫能設計方能有效與安全的應用。可以燃料電池或氫引擎等應用於車輛、家庭用電、工廠用電等。
\subsubsection{太陽能發電}
抵達地球的太陽能僅約一萬五千分之一即可滿足人類需求。透過太陽能板蒐集光能轉化成電能的過程。太陽能板主要由高純度的矽製成,傳統太陽能電池是使用矽晶元半導體的發電裝置,其原理是陽光照射在太陽能板上,電池吸收300-1100nm波長的陽光,發生光伏效應,產生直流電的發電方式。目前太陽能光電板轉換效率最高約為25\%,安全性亦高,但成本較高,面積能量密度較低,且受天候影響,多未達經濟效益。太陽能板退役後不可回收,且生產過程中會造成汙染,且會犧牲耕作面積。
\subsubsection{太陽能熱水器}
由集熱器、熱水儲熱筒、管路與自動控制系統組成。集熱器吸收太陽輻射熱,熱水儲熱筒儲存該等熱能,一般多用不銹鋼材質,再加上一層或數層絕熱材料,可保持水溫一天下降3℃以內。一般若有瓦斯、天然氣或電能熱水器,可將管線與太陽能熱水系統的管線做串聯或並聯,可在水溫不夠高又急需使用熱水時滿足需求,亦能節省部分燃料費用。大部分家用太陽能熱水器都裝有輔助電熱器,可在日照不足的情況下自動啟動。大型強制式太陽能熱水系統之儲熱筒與集熱器間裝有溫差控制器,當兩者溫差達一定程度時,自動啟動泵浦將儲熱筒內較冷之水打至集熱器吸收太陽輻射能後再流回儲熱筒。
\subsubsection{風力發電}
利用風力帶動風車葉片旋轉,使渦輪發電機發電。新式風力發電機的大型風車葉片使用玻璃纖維製成,密度低、強度高,每秒3公尺的二級風即可發電,但穩定度低,風力過強如颱風會使發電機毀損不能使用,且可能遭鳥擊、雷擊、鹽害,若風速不穩定或風向經常改變,也會降低發電效率。成本仍高於火力發電,且陸域風力發電有低頻噪音汙染,而離岸風力發電則干擾海洋生態,如施工噪音影響附近的鯨豚,施工時可用氣泡幕降低傷害。臺灣海峽風速強而穩定,適合之,目前臺灣風力發電占比仍低於1\%。歐洲許多國家大量使用。
\subsubsection{生質能}
指含有機物的生物質經直接或間接轉換成能源使用,略可視為零碳排放增加,可用於發電、交通運輸等。主要原料包括:
\begin{itemize}
\item \tb{農林作物}:如甘蔗、玉米、菜籽油、稻桿、麥桿、木屑、木材、醣類(如纖維素、澱粉、蔗糖)作物、油脂作物。有與糧食競爭之質疑。植物纖維作為生質能發電的燃料又稱青草汽油。
\item \tb{民生與工業廢棄物、廢水與廢氣}。
\item \tb{海洋資源}:如藻類,如萊茵衣藻在特定情況下會釋放氫氣,稱生物氫。
\item \tb{沼氣(Marsh gas, swamp gas or bog gas)}:甲烷和少量硫化氫、二氧化碳和微量其他物質組成的氣相混合物,在一些沼澤、溼地、動物排泄物、有機廢水、垃圾與畜牧場等中產生。
\item \tb{動物脂肪}。
\end{itemize}
轉換方式如:
\begin{itemize}
\item 物理轉換:壓縮、乾燥、破碎等。
\item 化學轉換:碳化、液化、氣化等。
\item 生物轉換:發酵(製造乙醇、丁醇等)、消化等。
\end{itemize}
產物如:
\begin{itemize}
\item 固態:碎片、顆粒、碳磚等。
\item 液態:
\begin{itemize}
\item \tb{生質酒精}:主要利用醣類(如甘蔗、玉米、高粱、甘藷、甜菜、稻桿)發酵製成。
\item \tb{生質柴油}:生質柴油為酯類,而非如石化柴油為烷類。主要利用廢食用油、動物脂肪、油脂作物(如大豆、花生、葵花子、棕櫚樹、痲瘋樹)等轉酯化製成。
\item \tb{生質汽油}:汽油中添加10\%以下之無水酒精。可提高汽油燃燒效率並降低一氧化碳排放。
\end{itemize}
\item 氣態:沼氣、合成氣等。
\end{itemize}
生質能生產技術簡單,所需溫度不高,僅約5-35°C,但許多水分含量較多,可達50-95\%,轉換效率較低、面積能量密度較低。
\subsubsection{水力發電}
利用地面逕流或水庫的水位落差,以水流推動水輪機,水輪機連接渦輪發電機,使重力位能轉換為動能再轉換為電能。水力是最古老的能源之一,也是再生能源發電中最具經濟效益者。水壩可防洪、蓄水,可儲存性高,但會影響生態且受地理限制。臺灣因地形限制、河川短促、雨量不均,水力發電量低。
\subsubsection{潮汐發電}
類似水力發電,利用漲退潮的位能差推動渦輪運轉而產生電能,但潮差達6m以上才具經濟效益。臺灣本島無,金門、馬祖或許可。
\subsubsection{溫差發電}
以海水表層與深層溫差發電。海洋表層的海水溫度高,可使低沸點的流體蒸發成氣體來推動渦輪運轉,使用後的蒸氣再以深層的海水冷卻成液體,反覆運作,可發電,目前無商業運轉者。臺灣南部夏季海水溫度高達30°,溫差大,深具潛力;東部地區地形陡峭,離岸不遠即達800公尺深以上,又有黑潮通過,溫差經年在20°C左右,深具潛力,惟有地層滑動之疑慮,臺灣東部廠址已初期評估。
\subsubsection{洋流/海流發電}
利用洋流/海流的動能推動渦輪機運轉產生電能,雖然洋流/海流易預測,但技術未成熟、設備成本高、輸電線路長且維護困難,又若大規模運用可能改變洋流系統。臺灣東部外海有黑潮通過,或可行之。
\subsubsection{波浪發電}
設施於海面上,中一室,上有而二活塞分別可使可空氣進與出,中為渦輪機,波浪上下往返運動產生壓力改變室中氣壓,上時增壓使空氣自活塞出去,下時減壓使空氣自活塞進入,推動渦輪產生電能,是目前最成熟的海洋能技術。設備工程難度高,須忍受海水侵蝕。臺灣東北部波浪大,有多處適合場址,未來或將發展,但岸基型易受颱風破壞,離岸型技術門檻較高。
\subsubsection{地熱發電}
使用地熱能發電。陸地地熱約一半來自放射性核種衰變放熱,海洋地熱約85\%來自海洋板塊冷卻放熱。將水灌入地底,使受熱蒸發,再從地底導出高溫水蒸氣推動渦輪發電機。地熱廠建造快速容易,熱效率約30\%,自地底導出的熱水通常酸性強,含二氧化硫等硫化物,腐蝕性強,且氨、二氧化硫等有毒氣體可能噴入空氣中。臺灣有優良地熱能條件但尚難克服腐蝕性問題。


\section{空氣汙染(Air pollution)}
\ssc{汙染源與防治}
\subsubsection{自然汙染源}
火山爆發、森林大火產生的二氧化硫、二氧化碳、春夏的大量花粉等,多局部且短暫,且大氣有自我淨化能力,多不會有嚴重影響。
\subsubsection{人為汙染源}
發電廠、工廠、燃油車等所排放的有害氣體,因具有區域性及持續性,導致汙染物排放量過多,造成自然不及自我淨化。
\subsubsection{各空氣汙染物的主要來源}
\begin{itemize}
\item\tb{碳氧化物}:化石燃料燃燒、內燃機廢氣、分解作用、呼吸作用、海洋氣體交換。
\item\tb{硫氧化物}:化石燃料等中的硫化物燃燒或遇水、硫酸、金屬冶煉等工廠廢氣。
\item\tb{氮氧化物}:燃油車內燃機高溫時空氣中的氮氣與氧氣反應、工廠廢氣、閃電。
\item\tb{懸浮微粒與細懸浮微粒}:工廠廢氣、燃燒燃料、金紙、香、菸、垃圾的煙霧、沙塵暴、森林大火、漂浮海鹽、細菌。
\item\tb{氟氯碳化物、氫氟氯碳化物、氫氟碳化物、全氟碳化物}:噴霧劑和冷媒/製冷劑。
\end{itemize}
\sssc{各空氣汙染源的防治}
\begin{itemize}
\item \tb{燃油車廢氣}:燃油車排氣管的觸媒轉化器中,鉑、銠、鈀或五氧化二釩等作為非勻相光觸媒,將廢氣中的碳氧化物、碳氫(氧)化物、氮氧化物轉化成水、二氧化碳與氮氣釋放。
\item \tb{工廠廢氣的懸浮微粒與細懸浮微粒}:工業排氣先經靜電集塵器處理。
\item \tb{工廠的含硫廢氣}:使用含硫量低的燃料、大型煉油廠與火力發電廠等排氣前先做脫硫處理,排放含硫廢氣前添加鹼性溶液或石灰,石灰吸收硫氧化物反應式:
\[\ce{2CaO(s) + 2SO2(g) + O2(g) -> 2CaSO4(s)}\]
\[\ce{CaO(s) + SO3(g) -> CaSO4(s)}\]
\item \tb{工廠的含氮廢氣}:在催化劑存在與高溫下,利用氨或尿素等氨的衍生物還原氮氧化物為氮氣。
\item \tb{農牧業的氮肥過度使用}:合理化施肥與推廣有機質肥料,減少化肥的過度使用,監測養殖場。
\item \tb{冷煤與噴霧劑的氟氯碳化物、氫氟氯碳化物、氫氟碳化物、全氟碳化物}:使用替代品,如環保冷媒。
\item \tb{民生廢氣}:使用環保金紙、金紙集中燃燒、減少燃燒金紙與香、減少抽菸、以大眾運輸或電動車替代燃油車。
\end{itemize}
\ssc{空氣汙染的種類}
\subsubsection{酸雨(Acid rain)}
\begin{itemize}
    \item \textbf{定義}:雨水 pH<5.0。
    \item \textbf{成因}:雨水原本因溶解二氧化碳成碳酸呈 pH5.5-6.0,化石燃料燃燒排放的硫氧化物溶於雨水成硫酸或亞硫酸、氮氧化物溶於雨水成硝酸或亞硝酸,使其更酸。
    \item \textbf{影響}:
    \begin{itemize}
        \item 腐蝕含碳酸鹽的建材。
        \item 因難於溶於水而易溶於酸的地表重金屬溶出,影響水質與土壤並進入食物鏈富集。
        \item 將土壤中的離子化合物帶往深層沉積,使淺根系植物無法吸收足夠養分,甚至森林枯萎。
        \item 含硫與含氮空氣與雨水可能引起呼吸道過敏、刺激眼睛與皮膚。
        \item 水域酸化,魚蝦死亡。
    \end{itemize}
\end{itemize}
\subsubsection{溫室效應(Greenhouse effect)與氣候變遷(Climate change)}
\begin{itemize}
    \item \textbf{定義}:太陽短波輻射約47\%被地表吸收,地表再以紅外線輻射而出。大氣中的溫室氣體吸收來自地表的紅外線輻射並再放射紅外線輻射,稱溫室效應,可讓地球均溫維持在15°C。但溫室氣體過多,導致地球表面溫度上升,至2015年約上升0.75攝氏度。溫室氣體中,水約占整體溫室效應6到7成,二氧化碳(\ce{CO2})百分之二十六,其餘有氧化亞氮(\ce{N2O})、甲烷(\ce{CH4})、六氟化硫(\ce{SF6})、三氟化氮(\ce{NF3})、氫氟碳化物(HFCs)、全氟碳化物(PFCs)、氟氯碳化物(CFCs)、氫氟氯碳化物(HCFCs)等。
    \item \textbf{成因}:太陽輻射主要是短波輻射,地面輻射和大氣輻射則是長波輻射。大氣對長波輻射的吸收力較強,對短波輻射的吸收力較弱。當太陽光照射到地球上,部分能量被大氣吸收,部分被反射回宇宙,大約47\%的能量被地球表面吸收,同時地球表面無論晝夜都以紅外線的方式向宇宙散發吸收的能量,其中也有部分被大氣吸收。大氣層保存了一定的熱量,並與地球互相輻射與吸收紅外線。一些理論認為,由於溫室氣體的增加,使地球整體所保留的熱能增加,導致全球暖化。
    \item \textbf{影響}:
    \begin{itemize}
        \item 全球氣溫上升,氣候變遷。
        \item 兩極、格陵蘭與高山冰原與冰川融化,海平面上升。
        \item 寒帶與高山物種滅絕、物種棲息範圍向高緯度移動。
        \item \tb{極端天氣(Extreme weather)}:指非正常、嚴重、季節性,或超出歷史平均數值的天氣。通常極端天氣以當地過往天氣數字為基礎,並被訂為是基數中最不常見的百分之十。例如:2012至2016年加州乾旱,而後面臨暴雨與暴雪,2020至2022年又乾旱;2023年澳洲經多年反聖嬰現象後進入聖嬰現象,發生複合火災、洪水、破紀錄熱浪、暖冬、極端降雨、嚴重風暴事件。
    \end{itemize}
\end{itemize}
\subsubsection{臭氧層破洞(Ozone hole)}
\begin{itemize}
    \item \textbf{定義}:約距地表20-30km的臭氧層內的臭氧濃度顯著降低,根據臭氧總量衛星觀測儀器(TOMS)的測量,發現地球平流層的臭氧自1970年起以每10年約4\%速率遞減,尤其兩極部分季節遞減速度更快,稱臭氧層破洞。
    \item \textbf{成因}:過去用於空調、冰箱的舊式冷煤與過去噴霧劑推進劑中的氟氯碳化物(CFCs)如 \ce{CF2Cl2}、\ce{CFCl3} 等在對流層中相當穩定,但進入平流層後在紫外線作用下分解出氯原子,成為臭氧分解的催化劑。另外,自由基 \ce{NO} 亦會造成之。
    \item \textbf{影響}:增加到達地表的紫外線輻射,對動物皮膚、眼睛和免疫系統有害,增加人類皮膚癌和白內障的風險,造成植物生長遲滯,對生態系統和農作物生產造成損害。
\end{itemize}
\subsubsection{懸浮微粒(Particulate matter, PM)}
\begin{itemize}
\item \textbf{定義}:懸浮在空氣中的固體顆粒或液滴稱懸浮微粒,直徑小於等於10微米者稱可吸入懸浮微粒 PM10,直徑小於等於2.5微米者稱細懸浮微粒 PM2.5。直徑大於10微米多會重力沉降,形成落塵而不是懸浮微粒。懸浮微粒的成分主要有硫酸鹽、硝酸鹽、銨鹽、氯化鈉、鉛、無定形碳、礦物粉塵、重金屬、氨、水等,上可附著硫氧化物、氮氧化物、烴類等。
    \item \textbf{影響}:
    \begin{itemize}
        \item 懸浮微粒約為細菌大小,上易吸附多環芳香烴等有機汙染物、重金屬等,能進入支氣管壁,干擾肺部氣體交換,被吸入肺泡亦對呼吸系統傷害大,造成肺功能下降、氣喘惡化、胎兒高血壓與出生缺陷、呼吸系統和心血管系統疾病等,PM2.5 較小者可能誘發癌症。
        \item 霾害減少能見度,影響交通安全。
        \item 戴口罩可阻擋多數。
    \end{itemize}
\end{itemize}
\subsubsection{光化學煙霧(Photochemical smog)}
\begin{itemize}
    \item \textbf{定義}:在陽光下經光化學反應形成的二次汙染物,如氮氧化物、烴類與對流層臭氧,主要來自氮氧化物和揮發性有機化合物(Volatile organic compounds, VOCs)。
    \item \textbf{成因}:
\begin{itemize}
\item 一次汙染物(指由汙染源直接產生者)經紫外線照射發生光化學反應,形成二次汙染物。
\item 臭氧由碳氫化合物與氮氧化合物發生光化學反應形成。
\item \ce{NO}和\ce{O2}形成紅棕色\ce{NO2},易於盆地城市或對流層低空逆溫處形成粉紅色光化學煙霧。
\end{itemize}
\item \textbf{影響}:
\begin{itemize}
\item 未飽和烴、醛等產生有毒物質,造成人體呼吸、排泄、神經、造血系統疾病、致癌、胎兒畸形。
\item 減少能見度,影響交通安全。
\end{itemize}
\eit
\subsubsection{使血紅素失去運送氧氣的能力}
\textbf{定義}:\ce{CO}、\ce{NO} 進入體內後會與血紅素結合,使失去攜氧能力,造成細胞缺氧而窒息死亡。
\subsection{空氣品質與其檢測}
\subsubsection{紫外線吸收法(UV Photometric Method)測量臭氧濃度}
臭氧能夠吸收254 nm的紫外光,其吸收強度與濃度成正比,故以紫外光臭氧分析儀(UV Ozone Analyzer)測量紫外光強度,再以比爾-朗伯定律(Beer-Lambert Law)計算臭氧濃度。
\subsubsection{β 射線吸收法(Beta Attenuation Method, BAM)測量懸浮微粒濃度}
以碳-14或氫-3等發射 β 射線,使穿過收集了微粒的濾紙,測量衰減量來計算濃度。
\subsubsection{光散射法(Light Scattering Method)測量懸浮微粒濃度}
使用特定光源照射懸浮粒子,測量散射光強度來估算粒徑與數量。
\subsubsection{質量慣性法懸浮微粒濃度}
濾紙裝置在天平上,其後有排氣管路,利用微量天平量測濾紙重量,當氣體樣品通過濾紙時,濾紙會因收集了空氣中的懸浮微粒而增加質量,由此換算懸浮微粒濃度。
\subsubsection{非分散式紅外線吸收法(Non-Dispersive Infrared Absorption, NDIR)測量一氧化碳濃度}
一氧化碳可吸收約4.6 µm紅外線,利用非分散紅外線吸收光譜技術測量CO的濃度。
\subsubsection{紫外螢光法(UV Fluorescence Method)測量二氧化硫濃度}
二氧化硫被約214 nm紫外光激發後,會在電子躍遷回基態時發射螢光,透過檢測螢光強度來計算二氧化硫濃度。
\subsubsection{化學發光法(Chemiluminescence Method)測量氮氧化物濃度}
一氧化氮與臭氧反應產生激發態的二氧化氮,躍遷回基態時發出的光稱化學發光(Chemiluminescence),檢測化學發光強度來計算氮氧化物濃度。
\subsubsection{空氣品質指標(Air Quality index, AQI)}
依據監測資料將當日空氣中臭氧、細懸浮微粒、懸浮微粒、一氧化碳、二氧化硫及二氧化氮濃度等數值,以其對人體健康的影響程度,分別換算出不同汙染物之副指標值(測值與副指標值為線性關係),再以當日各副指標之最大值為該測站當日之空氣品質指標值。6等級分別是良好、綠,0-50;普通、黃,51-100;對敏感族群不良、橘,101-150;對所有族群不良、紅,151-200;非常不良、紫,201-300;有害、褐紅色,301-500。
\sssc{全球暖化潛勢(Global warming potential, GWP)}
衡量溫室氣體在特定時間內相對於二氧化碳在大氣中吸收的熱量,以二氧化碳為一。甲烷約 25、一氧化二氮約 298。


\section{水汙染(Water pollution)}
\ssc{水汙染的種類}
\subsubsection{優養化(Eutrophication)/植物營養物質汙染}
\begin{itemize}
    \item \textbf{定義}:水體中的植物營養物質(如氮、磷)過剩,導致藻類和其他水生植物過度生長,造成水體溶氧量下降,水質惡化,動植物衰亡。
    \item \textbf{成因}:農業和城市排放的含有氮、磷的廢水流入水體,如含有磷酸鹽的肥料、殺蟲劑、清潔劑。
    \item \textbf{影響}:
    \begin{itemize}
        \item 藻華現象,水體變綠,影響水質。
        \item 水體中的氧氣含量降低,造成魚類和其他水生生物死亡。
    \end{itemize}
\end{itemize}
\subsubsection{缺氧/需氧廢料汙染}
\begin{itemize}
    \item \textbf{定義}:水體中的需氧廢料(如有機物)被細菌分解消耗水中溶氧。
    \item \textbf{成因}:家庭、農業、工廠廢水、動植物廢料。
    \item \textbf{影響}:
    \begin{itemize}
        \item 生物缺氧死亡。
        \item 骯髒與臭味。
    \end{itemize}
\end{itemize}
\sssc{傳染媒介汙染}
\begin{itemize}
    \item \textbf{定義}:水體中含有傳染病傳播媒介,如細菌、病毒、帶病生物。
    \item \textbf{影響}:傳染病。
\end{itemize}
\subsubsection{有毒無機化學品汙染}
\begin{itemize}
    \item \textbf{定義}:水體中含有有毒無機化學品。重金屬者稱重金屬汙染。
    \item \textbf{成因}:工業排放進入水體,如研磨廢水、酸性廢水與電鍍廢水,及地下水原有。常見有毒無機化學品汙染源:
\begin{itemize}
\item 汞:鹼氯工業、電化學工業、殺菌劑。
\item 鎘:電鍍工業、含鋅產品雜質。
\item 鉛:含鉛石油產品、顏料、油漆、器皿。
\item 氰:金屬表面處理工業。
\item 砷:含砷礦的地下水、煤炭燃燒雜質、磷酸鹽雜質。
\item 鈹:煤炭燃燒雜質、核子燃料、火箭燃料。
\item 銅:五金工廠、晶圓廠。
\item 氟:晶圓廠。
\item 氯:除草劑、農藥。
\end{itemize}
    \item \textbf{影響}:通過生物富集作用累積,影響生物生理機能。許多重金屬離子僅需痕量(trace)即產生毒性,一般重金屬離子須 1 mg/L 以上,但鎘、汞僅須 0.001 mg/L 以上。某些金屬離子如汞離子、二價鉛離子等,與生物體內酵素之活性中心結合,使失去活性。常見有毒無機化學品對人體的影響:
\begin{itemize}
\item 汞:水俁病、腦損傷,常致死。
\item 鎘:低濃度(2 mg/L)造成高血壓,高濃度造成痛痛病(軟骨症)。
\item 鉛:低劑量使人急躁、噁心,高劑量造成神經、平滑肌、腎臟、生殖功能障礙。
\item 氰:劇毒,常致死。
\item 砷:大量服用造成腸胃障礙、小量長期服用致癌、烏腳病。
\item 銅:肝硬化、知覺神經受損、運動障礙與消化系統傷害。
\item 鈹:小量長期服用致肺損傷,常致死。
\end{itemize}
    \item \textbf{事例}:
    \begin{itemize}
    \item 1950日本富山縣市因山區礦場排放鎘入河發生鎘中毒事件,因患者關節和脊骨極度疼痛稱痛痛病。
    \item 1950臺灣西南沿海居民因飲用含高量砷的深井水而發生末梢血管阻塞疾病,因患者雙足發黑稱烏腳病。
        \item 1956日本九州水俁(yŭ)市鹼氯工廠將含汞廢水排放至海灣,造成甲基汞中毒事件,使上萬人罹患水俁病。
        \item 1982臺灣桃園縣觀音鄉工廠廢水含鎘發生鎘米事件。
        \item 1986臺灣臺南、高雄二仁溪兩岸五金業者將廢液傾倒溪中,使海口處的養殖牡蠣經生物累積作用(bioaccumulation)大量吸收銅離子,顏色變綠,稱綠牡蠣(green oyster)事件。
        \item 1997臺灣香山區疑因園區廢水汙染出現綠牡蠣。
    \end{itemize}
\end{itemize}
\subsubsection{泡沫汙染/硬性界面活性劑汙染}
\begin{itemize}
    \item \textbf{定義}:水體中出現大量穩定的泡沫,主要由於硬性界面活性劑的存在。
    \item \textbf{成因}:家庭和工業排放的洗滌劑、清潔劑等含有硬性界面活性劑。
    \item \textbf{影響}:泡沫覆蓋水面,影響水生生物的呼吸和光合作用。
\end{itemize}
\subsubsection{熱汙染}
\begin{itemize}
    \item \textbf{定義}:水體溫度異常升高,通常由於人類活動引起。
    \item \textbf{成因}:化工廠冷卻水排放、發電廠排放熱水等。
    \item \textbf{影響}:降低水中的溶解氧含量,影響水生生物的生存,造成熱病變,促進有害藻類和微生物的繁殖,造成珊瑚白化。
    \item \textbf{防治}:採用冷卻塔或冷卻池降低工業冷卻水的溫度。
\end{itemize}
\subsubsection{酸化}
\begin{itemize}
    \item \textbf{定義}:氫離子濃度上升。
    \item \textbf{成因}:工廠二氧化硫與酸性廢水等的排放、大氣與海洋氣體交換。
    \item \textbf{影響}:鈣化作用降低,具碳酸鈣質殼骨的生物,如珊瑚、貝類,殼體變薄。
\end{itemize}
\subsubsection{懸浮物汙染}
\begin{itemize}
    \item \textbf{定義}:水體中存在懸浮物,如塑膠微粒。
    \item \textbf{成因}:廢棄物丟棄、土地侵蝕。
    \item \textbf{影響}:塑膠微粒等被生物誤食,經生物富集作用,影響生物生理機能。
\end{itemize}
\subsubsection{放射性物質汙染}
\begin{itemize}
\item \textbf{定義}:水體中存在放射性物質。
\item \textbf{成因}:放射性落塵、放射性廢料,如核電廠之核廢料。
\item \textbf{影響}:游離輻射影響生物生理機能。
\end{itemize}
\subsection{水質與其檢測}
\sssc{懸浮固體物(Suspended solids)}
水中會因攪動或流動而呈懸浮狀態之固體物。單位毫克每公升。

中華民國農業部灌溉用水標準為 100 mg/L 以下,中華民國農業部畜牧業流放水標準為 150 mg/L 以下。
\subsubsection{電極法測量導電度/電導率((Electrical) conductivity, EC)}
通過單位截面積單位高度液柱的電阻的倒數,單位微西門子每公分 (μS/cm) = 微姆歐每公分 (μmho/cm)。當溶液中離子與導電物質愈少則導電度愈低。固體溶質多溫度愈高解離常數愈大。

導電度計須使用 0.01 M \ce{KCl(aq)} 為標準溶液進行校正,且電極上不可附著物質。由於溫度會影響測量的電位,若測計無溫度補償,須同時測量溫度,以進行溫度補償操作。

中華民國自來水約 350 μS/cm,中華民國農業部灌溉用水標準為 750 μS/cm 以下。
\sssc{pH計(pH meter)/酸鹼測計}
pH 計利用玻璃電極及參考電極的電位差以電極法測量樣品質子濃度。

pH 計須兩點校正,先以 pH 7.0$\pm$0.5 之中性緩衝溶液進行零點校正,再以相差 2 至 4 個 pH 值之酸性或鹼性緩衝溶液進行斜率校正,此二校正點之區間移涵蓋待測樣品 pH 值。由於溫度會影響測量的電位,若測計無溫度補償,須同時測量溫度,以進行溫度補償操作。
\subsubsection{電極法測量溶氧(Dissolved Oxygen)度/DO 值}
液體中溶解的氧氣濃度。評估河川水質時主要指標之一,水質愈佳,溶氧通常愈接近飽和。溶氧是水中好氧生物賴以維生之物。需氧廢料(如有機物、營養鹽)之分解耗去水中溶氧。氧的溶解度與溫度負相關,20°C 1 atm 約 9.1 mg/L。

溶氧電極利用選擇性透膜,讓水中溶氧在陰極還原成水,由半反應產生之電流強度正比於水中溶氧換算出水的 DO 值。溶氧電極須兩點校正,一為零溶氧試劑的零點,另一為飽和溶氧試劑的滿點。

中華民國農業部灌溉用水標準為 3 mg/L 以上,中華民國環境部水質標準河川溶氧小於 2 mg/L 其受有機物汙染程度為嚴重。
\subsubsection{生化需氧量(biochemical oxygen demand, BOD)與化學需氧量(chemical oxygen demand, COD)}
\begin{itemize}
    \item \tb{生化需氧量(biochemical oxygen demand, BOD)}:指五日內水中微生物分解有機物時所需的氧氣量。單位毫克每公升。通過將水樣在20°C恆溫培養箱暗處培養5天後測定消耗的溶氧量得知。中華民國農業部畜牧業流放水標準為 80 mg/L 以下。
    \item \tb{化學需氧量(chemical oxygen demand, COD)}:指在水中加入酸性\ce{K2Cr2O7}或酸性\ce{KMnO4}等強氧化劑,氧化有機物所需的等效氧氣量。二鉻酸鉀用量可用硫酸鐵(II)銨\ce{(NH4)2Fe(SO4)2}滴定得。單位毫克每公升。中華民國農業部畜牧業流放水標準為 450 mg/L 以下。
    \item BOD$\leq$COD,其中當水中需氧廢料不能完全被微生物分解時BOD<COD。
\end{itemize}
\subsubsection{光散射法測量濁度(Turbidity)}
衡量對光源的散射能力,單位 Nephelometric turbidity unit,簡稱 NTU 或度,定義為 1 公升的水中含有 1 毫克的二氧化矽時的水樣的混濁程度。

光散射法濁度檢測的原理為在特定條件下比較待測水樣和參考濁度懸浮液對特定光源的散射光強度,散射光愈強,濁度愈大。

濁度計使用前依製造商提供之儀器校正手冊校正。測定時,先搖動待測水樣使懸浮固體顆粒均勻分散,待氣泡消失後,將水樣倒入試管中,自濁度計讀取測量值。


\section{循環}
\subsection{水循環 (Water cycle)/水文循環(Hydrologic/Hydrological cycle)}
\subsubsection{水交換}
\begin{itemize}
\item \tb{平流}:水在大氣中移動。
\item \tb{凝結}:水蒸氣變成水滴。
\item \tb{凝華}:水蒸氣變成冰。
\item \tb{蒸發散}:液態水蒸發為氣態(含植物蒸散作用),造成蒸發的能量來源主要是太陽輻射,地球每年蒸發散的水總量估計約為505000立方公里,其中86\%源自海洋。
\item \tb{滲透}:水從地表滲透進入地下,變成土壤水分或地下水。
\item \tb{攔截}:受到植物攔截而未落到地面的降水。
\item \tb{滲流}:水因重力作用,垂直往下穿過土壤和岩石的作用。
\item \tb{降水}:液態水沉降到地表,大多數以雨的形式出現,也包括雪、冰雹、霧滴、霰等。全球每年大約有505000立方公里的降水,其中78\%發生在海洋上。
\item \tb{地表徑流}:水在陸地上流動。
\item \tb{融化}:積雪與冰川等冰融化成水。
\item \tb{昇華}:冰變成水蒸氣。
\item \tb{地下徑流}:水在地下滲流帶和含水層中流動。接著可能會返回地表(如湧泉或受抽取)或滲入海洋。地下水通常移動緩慢也補充緩慢。
\end{itemize}
\subsubsection{水分於儲藏處平均停留時間}
\begin{longtable}[c]{|c|c|}
\hline
儲藏所在 & 平均停留時間\\\hline\endhead
南極洲 & 20000年\\\hline
海洋 & 3200年\\\hline
冰河 & 20到100年\\\hline
季節性降雪 & 2到6月\\\hline
土壤水分 & 1到2月\\\hline
淺層地下水 & 100到200年\\\hline
深層地下水 & 10000年\\\hline
湖泊 & 50到100年\\\hline
河流 & 2到6月\\\hline
大氣 & 9天\\\hline
\end{longtable}\FB
\subsubsection{全球水分布}
\begin{itemize}
\item \tb{鹹水}:97.47\%
\item \tb{淡水}:2.53\%
\begin{itemize}
\item \tb{冰帽和冰川}:68.7\%
\item \tb{地下水(Groundwater)}:30.1\%
\item \tb{地面冰和永凍土(Permafrost)}:0.8\%
\item \tb{地表水}:0.3\%
\begin{itemize}
\item \tb{淡水湖泊}:67.4\%
\item \tb{土壤水分(Soil moisture)}:12.2\%
\item \tb{大氣水}:9.5\%
\item \tb{溼地(Wetlands)}:8.5\%
\item \tb{河流}:1.6\%
\item \tb{生物}:0.8\%
\end{itemize}
\item \tb{其他}:0.1\%
\end{itemize}
\end{itemize}
\subsection{碳循環(Carbon cycle)}
\bct\bfH\ctr\icg[width=0.9\textwidth]{carbon_cycle.jpg}\ef\FB\ect
\subsubsection{碳匯/碳庫(Carbon sink)}
儲存碳之處。地球主要碳匯包含:
\begin{itemize}
\item \tb{大氣圈}:主要儲存形式為二氧化碳,次之為甲烷。過去兩個世紀中的人類活動已導致大氣中的碳含量增加近50\%。儲存量約\scinote{720}{12}公斤。
\item \tb{陸地生物圈}:囊括所有陸地上的生物與死的生物質中的有機碳,包含儲存於土壤中的碳。儲存量約\scinote{2000}{12}公斤,其中約\scinote{800}{12}公斤在活的生物質,\scinote{1200}{12}公斤在死的生物質。
\item \tb{水生生物圈}:約$1$至\scinote{2}{12}公斤。
\item \tb{岩石圈}:碳循環運作緩慢,碳主要以惰性方式儲存在岩石中。其中部分是以有機碳的形式從生物圈沉積而來,經過高溫與高壓的沉積和埋藏形成油母質。沉積碳酸鹽約\scinote{60}{18}公斤,油母質約\scinote{15}{18}公斤。
\item \tb{海洋圈}:表面層中的溶解無機碳(Dissolved inorganic carbon, DIC)與大氣快速交換,維持平衡。深層 DIC 濃度較表面層高出約15\%,且體積較大,含有更多的碳,是世界上最大的主動循環碳庫,與大氣達到平衡的時間尺度需要數百年。
\end{itemize}
\subsubsection{碳交換}
\begin{itemize}
\item \tb{溫鹽環流}:海洋表面層$\leftrightarrow$深層。
\item \tb{光合作用}:大氣圈$\rightarrow$生物圈。
\item \tb{呼吸作用}、\tb{分解作用}:生物圈$\rightarrow$大氣圈、海洋圈。
\item \tb{大氣中的氧溶於海中與反之}:大氣圈$\leftrightarrow$海洋表面層。
\item \tb{風化作用}:岩石圈$\rightarrow$大氣圈、海洋圈。
\item \tb{燃燒}:岩石圈$\rightarrow$大氣圈,近年來因人類活動大幅增加,使大氣溫室氣體濃度增加。
\item \tb{成岩作用}:生物圈、大氣圈、海洋圈$\rightarrow$岩石圈。
\end{itemize}
\sssc{碳捕獲與封存(Carbon capture and storage, CCS)/碳捕獲、利用與封存(Carbon capture, utilization, and storage, CCUS)}
指自二氧化碳源(如工廠)捕獲二氧化碳並運輸至長期封存地點(通常是深層地質構造或鹹水水域)封存的過程。

現行主要捕獲技術為令氣流通過具有胺基的聚合物或溶劑,再經高溫釋放。獲得的二氧化碳可淨化後用於製造乾冰等,或注入深層鹽水層、地底煤床、貧瘠油田、海洋等。
\subsection{氮循環(Nitrogen cycle)}
\subsubsection{固氮作用(Nitrogen fixation)}
指將氣態氮轉變為可被有機體吸收的氮化合物的過程。
\begin{itemize}
\item 一部分氮由閃電所固定,再溶於雨水中形成硝酸與亞硝酸落下:
\[\ce{N2(g) + O2(g) -> 2NO(g)}\]
\[\ce{2NO(g) + O2(g) -> 2NO2(g)}\]
\[\ce{2NO2(g) + H2O(l) -> HNO2(aq) + HNO3(aq)}\]
\item 大部分的氮被固氮細菌所固定,這些細菌擁有固氮酶,可催化氮氣氫化成為氨,生成的氨再被轉化為其他有機化合物,如在豆科植物的根瘤中與之共生的根瘤菌。
\item 近年來人類利用哈柏法(Haber process)大量製造氨,進而轉化為硝酸根離子或銨根離子等形式,製成肥料並加入土壤中,大大提升農作物的產量,並造成環境中硝酸根離子的累積速率大於其被反硝化的速率。
\end{itemize}
\subsubsection{同化作用(Assimilation)}
植物與浮游生物從土壤中與海水中吸收含氮離子如銨根離子和硝酸根離子,其中硝酸根離子通常先被還原為亞硝酸根離子,然後再還原為銨離子,這些含氮離子最後被轉化為其他有機含氮化合物,如胺基酸。
\subsubsection{氨化作用(Ammonification)}
細菌和真菌等分解者經由各種酶進行分解作用,將生物遺骸等中的有機含氮化合物轉化成氨。
\subsubsection{硝化作用(Nitrification)}
亞硝化將銨根離子轉化為亞硝酸根離子,硝化細菌再將亞硝酸根離子轉化為硝酸根離子。
\subsubsection{反硝化作用(Denitrification)}
反硝化細菌將硝酸根離子轉化為氮氣或一氧化二氮等,通常在厭氧環境進行。
\ssc{生物富集作用/生物放大作用(biomagnification or biological magnification)}
指自然環境中不易被生物排出的物質(如鉛、汞、DDT)在食物鏈(food chain)中疊加使濃度沿食物鏈在愈高級生物體內逐漸遞增的現象。


\section{環境與社會}
\subsection{環境相關國際條約簡史}
\sssc{蒙特婁議定書(Montreal Protocol)/關於消耗臭氧層物質的蒙特婁議定書(Montreal Protocol on Substances that Deplete the Ozone Layer)}
1987 年聯合國邀請所屬會員國在加拿大簽署,規定逐步淘汰氯氟碳化物(CFCs),使用替代品如氫氟碳化物(HFCs)、全氟碳化物(PFCs)等環保冷煤,2007第六次修正規定已開發國家和開發中國家分別應當在2030和2040年之前實現氫氯氟烴的完全淘汰。已具成效,根據美國太空總署,2009年臭氧破洞已不擴大,甚至開始修補。
\sssc{聯合國氣候變化綱要公約(United Nations Framework Convention on Climate Change, UNFCCC, FCCC)}
1992 年 155 個國家在聯合國環境與發展會議(The United Nations Conference on Environment and Development, UNCED)/地球高峰會(Earth Summit)/里約熱內盧高峰會(Rio de Janeiro Conference)簽署,屬綱要性公約,目標將大氣中溫室氣體的濃度穩定在防止氣候系統受到危險的人為擾動的水平上。這一水平應當在足以使生態系能夠自然地適應氣候變遷、確保糧食生產免受威脅並使經濟發展能夠永續地進行的時間範圍內實現。各國義務依共同但有差別的責任原則有所不同。如何具體進行溫室氣體減量則由公約議定書做詳細規定。
\sssc{京都議定書(Kyoto Protocol)}
1997 年在 FCCC 的基礎上簽署,目標將溫室氣體排放量減少到比1990年排放量低5.2\%,各國應減量多少依責任原則有不同,2005年生效。其規範對象僅限於 FCCC 的附件一國家,非附件一國家者如中國,印度,巴西,並不受拘束。而 FCCC 的附件一國家中亦有遲遲未批准者,如美國。
\sssc{關於持久性有機汙染物的斯德哥爾摩公約(Stockholm Convention on Persistent Organic Pollutants)}
2001 年簽署,2004 年生效,旨在禁止持久性有機汙染物(persistent organic pollutants, POPs)的生產與使用,禁止生產的物質如 1,2,3,4,5,6-六氯環己烷/六氯化苯(Benzene hexachloride, BHC)/加丹(Lindane)/蟲必死、多氯聯苯(Polychlorinated biphenyl, PCBs),限制生產的物質如1,1'-(2,2,2-三氯乙基-1,1-二亞基)雙(4-氯苯)/1,1,1-三氯-雙-2,2(4-氯苯基)乙烷/雙對氯苯基三氯乙烷(dichloro-diphenyl-trichloroethane)/滴滴涕(DDT)。
\sssc{哥本哈根協議(Copenhagen Accord)}
2009 年在哥本哈根的 2009 聯合國氣候變化大會(2009 United Nations Climate Change Conference)簽署,不具法律約束力。
\sssc{巴黎協定(Paris Agreement)}
2015 年在 FCCC 的基礎上簽署,目標把全球平均氣溫升幅控制在工業革命前水準以上低於2℃之內,並努力將氣溫升幅限制在工業化前水準以上1.5℃之內,同時認識到這將大大減少氣候變遷的風險和影響。提高適應氣候變化不利影響的能力並以不威脅糧食生產的方式增強氣候抗禦力和溫室氣體低排放發展。使資金流動符合溫室氣體低排放和氣候適應型發展的路徑。將減量排放溫室氣體的義務擴及至中國與印度等非附件一國家。由已開發國家提供綠色氣候基金,協助開發中國家執行減少溫室氣體排放與面對氣候變遷的調適。無設定強制約束力,依照目前的框架下由各國自主推動,每5年檢視減排成績。
\subsection{聯合國永續發展(Sustainable development)簡史}
\sssc{人類環境宣言(Declaration of the United Nations Conference on the Human Environment)/斯德哥爾摩宣言(Stockholm Declaration)}
1972 年瑞典聯合國人類環境會議(United Nations Conference on the Human Environment)中簽署,是國際條約中第一份提到人類有在健康生態環境下生活之權利的文件。
\sssc{布倫特蘭委員會(Brundtland Commission)/世界環境與發展委員會(World Commission on Environment and Development, WCED)}
1983 年成立的聯合國下屬組織。
\sssc{我們共同的未來(Our Common Future)/布倫特蘭報告(Brundtland Report)}
1987 年聯合國出版之刊物,闡述 WCED 的任務,明確定義永續發展為:「不損及後代子孫滿足其本身需求的發展,同時能滿足當代需求。」
\sssc{永續發展委員會(Commission on Sustainable Development, CSD)}
1993 年設置的聯合國下屬組織。
\sssc{千禧年發展目標(Millennium Development Goals)}
2000 年 CSD 推動,期待在2015年前實現該八大目標。
\sssc{永續發展目標(Sustainable Development Goals, SDGS)}
2015 年 CSD 推動,明定未來15年的17項經濟、社會與環境目標。
\subsection{美國}
\subsubsection{Clear Air Act}
\begin{itemize}
\item 六大重要汙染源:二氧化氮、臭氧、二氧化硫、微塵、一氧化碳和鉛。
\item 六大溫室氣體:二氧化碳、甲烷、一氧化二氮、氫氟碳化合物、全氟碳化合物和六氟化硫。
\end{itemize}
\subsection{中華民國}
\subsubsection{環境相關標誌}
\bct\bfH\ctr\icg[width=0.9\textwidth]{環保.jpeg}\ef\FB\ect\bct\bfH\ctr\icg[width=0.9\textwidth]{能源.jpg}\ef\FB\ect
\begin{itemize}
\item 節能標章:能源效率比國家認證標準高10\%至50\%。由電源插座、火苗與愛心雙手共同組成。我國特有。
\item 能源效率標示:年耗電量、能源等級標示、能源因數值等。我國特有。
\item 碳足跡標籤:代表該項活動或產品的整個生命週期(包含原料取得、製造、配送銷售、使用、廢棄回收階段)過程中直接或間接產生的溫室氣體排放總量,相當於排放多少二氧化碳的量所造成之溫室效應影響程度。我國者由碳足跡數字及計量單位、愛大自然的心、綠葉、 CO$_2$ Carbon Footprint Taiwan EPA 文字共同組成。
\item 省水標章
\item 綠建材標章
\item 環保標章
\item 碳手印:減碳行為
\end{itemize}
\subsubsection{國家氣候變遷調適政策綱領}
2012行政院提出國家氣候變遷調適行動方案及工作目標,包含8個調適領域:災害、維生基礎設施、水資源、土地使用、海岸、能源供給及產業、農業生產及生物多樣性與健康。
\subsubsection{永續發展政策綱領}
1997成立行政院國家永續發展委員會,訂立12大面向的永續發展指標系統架構。
\subsubsection{氣候變遷因應法}
氣候變遷因應法舊稱溫室氣體減量及管理法。

一、溫室氣體:指二氧化碳(CO2)、甲烷(CH4)、氧化亞氮(N2O)、氫氟碳化物(HFCs)、全氟碳化物(PFCs)、六氟化硫(SF6)、三氟化氮(NF3)及其他經中央主管機關公告者。

二、氣候變遷調適:指人類與自然系統為回應實際、預期氣候變遷風險或其影響之調整適應過程,透過建構氣候變遷調適能力並提升韌性,緩和因氣候變遷所造成之衝擊或損害,或利用其可能有利之情勢。

三、氣候變遷風險:指氣候變遷衝擊對自然生態及人類社會系統造成的可能損害程度。氣候變遷風險的組成因子為氣候變遷危害、暴露度及脆弱度。

四、溫室氣體減量:指減少人類活動衍生之溫室氣體排放或增加溫室氣體吸收儲存。

五、排放源:指直接或間接排放溫室氣體至大氣中之單元或程序。

六、溫暖化潛勢:指單一質量單位之溫室氣體,在特定時間範圍內所累積之輻射驅動力,並將其與二氧化碳為基準進行比較之衡量指標。

七、排放量:指自排放源排出之各種溫室氣體量乘以各該物質溫暖化潛勢所得之合計量,以二氧化碳當量表示。

八、負排放技術:指將二氧化碳或其他溫室氣體自排放源或大氣中以自然碳循環或人為方式移除、吸收或儲存之機制。

九、碳匯:指將二氧化碳或其他溫室氣體自排放源或大氣中持續移除後,吸收或儲存之樹木、森林、土壤、海洋、地層、設施或場所。

十、淨零排放:指溫室氣體排放量與碳匯量達成平衡。
\ssc{原則}
\sssc{5R 原則}
\begin{itemize}
\item \tb{減量(Reduce)}:減少廢棄物。
\item \tb{拒用(Refuse)}:拒絕使用高汙染或無法回收、再生、重複使用的物品。
\item \tb{重複使用(Reuse)}:重複使用物品。
\item \tb{回收(Recycle)}:回收未使用之物品。
\item \tb{再生(Regenerate)}:將廢棄物再生。
\end{itemize}
\sssc{原子(使用)效率/原子經濟(Atom economy)}
\[\frac{\tx{欲獲得的主要產物質量}}{\tx{所有產物總質量}}\]
\sssc{綠色化學(Green chemistry)12 項原則}
\begin{itemize}
\item \tb{防廢}:防止或減少廢物的產生。
\item \tb{簡潔}:最大化原子效率。
\item \tb{低毒}:設計低毒或無毒的化學方法。
\item \tb{保安}:設計安全的化學品,減少毒性的化學品。
\item \tb{降輔}:避免使用溶劑、混合物分離試劑和其他的輔助物質。如果必須使用,選擇無害的物質。
\item \tb{節能}:可能的話,在常溫常壓下進行反應。
\item \tb{再生}:使用可再生原料而非消耗型原料。可再生原料一般來源於農產品或是其他過程產生的廢物;消耗型原料一般來源於礦物或化石燃料等。
\item \tb{物盡}:減少或避免衍生反應,如保護基或其他暫時的修飾,因衍生步驟將使用額外的試劑並產生廢物。
\item \tb{催化}:盡可能選用催化劑。催化劑少量而可多次催化反應,優於一般過量且只能反應一次的當量試劑。
\item \tb{可解}:產物在功能結束後,應設計成可降解而不會在環境中累積。
\item \tb{監測}:即使監控以減少或消除有害副產物的生成。
\item \tb{思危}:設計化合物及其狀態等,以降低化學事故如爆炸、火災、洩漏的可能性。
\end{itemize}
\end{document}