\documentclass[a4paper,12pt]{article}
\setcounter{secnumdepth}{5}
\setcounter{tocdepth}{3}
\newcounter{ZhRenew}
\setcounter{ZhRenew}{1}
\newcounter{SectionLanguage}
\setcounter{SectionLanguage}{1}
\input{/usr/share/latex-toolkit/template.tex}
\begin{document}
\title{化學實驗}
\author{沈威宇}
\date{\temtoday}
\titletocdoc
\sct{化學實驗}
\ssc{試劑(Reagents)處理}
\sssc{燒杯(Beaker)}
圓柱形平底無蓋透明容器, 容量自數毫升至數公升不等,通常有杯嘴以便傾倒。主要用於混合、加熱、靜置液體等,不用於體積度量,加熱時宜間接加熱。
\sssc{錐形瓶(Conical flask)/三角燒瓶/滴定燒瓶(Titration flask)/鄂倫麥爾瓶(Erlenmeyer flask)}
\bit
\item 圓錐形身圓柱形頸平底透明容器。主要用於混合、加熱、靜置液體等,不用於體積度量,加熱時宜間接加熱。
\item 混合試劑而需加蓋時,用橡皮塞或軟木塞,勿使用手指。振盪時上下擺動。去蓋時需小心,振盪後壓力可能升高,試液可能噴出。
\eit
\sssc{平底燒瓶/佛羅倫薩燒瓶(Florence flask)/煮沸瓶(boiling flask)與圓底燒瓶(Round-bottom flask, round-bottomed flasks, or RB flasks)}
\bit
\item 平底燒瓶為圓球形身圓柱形頸平底透明容器;圓底燒瓶為圓球形身圓柱形頸圓底透明容器。
\item 主要用於混合、加熱、靜置液體等,不用於體積度量,加熱時宜間接加熱,加熱均勻,因圓球形身為各類燒瓶中最耐加熱而不易破裂者。
\item 混合試劑而需加蓋時,用橡皮塞或軟木塞,勿使用手指。振盪時上下擺動。去蓋時需小心,振盪後壓力可能升高,試液可能噴出。
\eit
\sssc{曲頸甑/曲頸瓶(Retort)}
一個球狀容器上有一開口向下的長窄頸,用於簡易蒸餾或乾餾,被餾液體置於球狀容器中並加熱,瓶頸相當於冷凝管,使蒸汽在其中冷凝,繼而順瓶頸流下,進入在下面放置的收集容器中。
\sssc{試管(Test tube, culture tube, or sample tube)}
\begin{itemize}
\item 細長指狀圓底透明容器。主要用於混合、加熱、靜置液體等。
\item 混合試劑而需加蓋時,用橡皮塞或軟木塞,勿使用手指。振盪時上下擺動。去蓋時需小心,振盪後壓力可能升高,試液可能噴出。
\item 在試管內混合試劑而無加蓋時,輕輕繞轉試管而攪動管內溶液,亦可使用玻棒攪拌溶液,但小心勿將管底打破。
\item 若部分固體附著於管壁,可輕敲管壁或以滴管吸取試管內的液體沖下。
\item \tb{試管夾(Test tube holder)}:有木製與金屬製者,可夾試管,之以利拿取。
\item \tb{試管架(Test tube rack)}:用於同時直立放置多個試管的實驗室設備。
\item \tb{試管刷(Test tube brush or spout brush)}:用於清潔試管和窄口器皿的細長刷子。
\end{itemize}
\sssc{廣口瓶}
圓柱形透明厚壁容器,常用於操作燃燒物質的實驗或作為集氣瓶,但不可以直接加熱。
\sssc{U 型管}
透明 U 型管,兩端開口,常用於乾燥、吸收、鹽橋等。
\sssc{研缽與杵/杵臼(Mortar and pestle)}
通常是瓷製的碗與杵,用來將固體磨成較細的粉末,也可以用來把不同的粉末混合。
\sssc{普通漏斗(Plain funnels)}
主要用於傾倒液體。
\sssc{長頸漏斗/薊頭漏斗(Thistle tube)}
管較長,常用於在會產生氣體的反應中添加液體,使用時須將末端置於液面以下,避免漏斗中的溶液因氣體進入而噴出。
\sssc{滴管(Eye dropper or dropper)/巴斯德吸量管(Pasteur pipette)}
管材如聚乙烯或玻璃,按壓處通常是橡皮頭,不可倒置以免試劑汙染橡皮頭。
\sssc{玻棒(Glass stirring rod, glass rod, stirring rod or stir rod)}
主要用於傾倒液體、攪拌等。
\sssc{錶玻璃(Watch glass)}
通常是圓形、淺盤狀的玻璃器皿。用途包括作為溶液蒸發時的容器、樣品容器、稱量時的盛皿,或者作為燒杯等的蓋子以防止塵土進入。
\sssc{橡皮塞(Rubber stopper)與軟木塞(cork stopper)}
用於塞住容器口,應使用適當大小的塞。部分有一或二孔,將細長物如溫度計、玻璃管、漏斗等插入橡皮塞或軟木塞之孔內時,須先用水或潤滑油沾抹各接觸表面,並用布包裹,再將之徐徐旋轉擠入,以免折斷。
\sssc{鐵架臺/支撐架/曲頸瓶架/蒸餾架(Retort stand, clamp stand, ring stand, or support stand)}
有一鑄鐵底座,上一側有一長桿旋入底座的螺紋,桿上可用滴定管夾、自由夾/萬用夾、彈簧夾、鐵夾等將實驗器具固定於其上以進行實驗。
\sssc{洗(滌)瓶(Wash bottle)}
頂上有鵝頸型噴嘴的瓶,常承裝蒸餾水用於洗滌容器。
\sssc{試劑保存}
\begin{itemize}
\item 易發生光化學反應的試劑,應裝在深棕色瓶內貯存,如硝酸水溶液與硝酸銀水溶液(以免釋出二氧化氮)、鹵素及其水溶液(尤其氯水,以免變成氫鹵酸與次鹵酸)、強氧化劑(如過錳酸鉀、雙氧水、二鉻酸鉀、氯酸鉀,以免釋出氧氣)。
\item 白磷應存放於水中,以免自燃。
\item 鋰、鈉、鉀、鈣、鍶、鋇須儲存在烷類中,如礦物油(mineral oil)/煤油(kerosene)/石蠟(paraffin)、石油醚(Petroleum ether)或甲苯,以免與空氣或水反應。
\item 玻璃耐酸不耐鹼與氫氟酸,不可將鹼性溶液長期存放於玻璃容器,強鹼、氫氟酸應存放於聚合物材質容器中。
\item 每一試劑瓶上貼標籤,標明試劑名稱和規格。
\item 應遠離熱源。
\end{itemize}
\sssc{試劑轉移與配置}
\begin{itemize}
\item \tb{粉狀固體}:粉狀固體試劑使用清潔之藥匙或刮勺挖取。
\item \tb{塊狀固體}:塊狀固體試劑使用清潔之鑷子夾取,放入易碎容器內時應沿內壁徐徐滑下。
\item \tb{塞或蓋}:瓶口隨開隨蓋,避免異物混入。塞或蓋放置時,與試劑接觸一面應朝上,不可接觸桌面或其他物品。不同藥品的塞或蓋不可混用。有手柄之細口瓶蓋可夾於食、中指間,並以該手持細口瓶倒出液體。
\item \tb{橡皮管}:套橡皮管時先用蒸餾水潤溼所有接觸面。
\item \tb{銅絲攪拌器}:將一根直徑約 1-2 mm 的銅絲末端用尖嘴鉗捲繞成孔徑約銅絲直徑 2-3 倍的圓圈,圓圈面與銅絲長柄垂直,可用於攪拌。
\item \tb{清潔}:使用前拭乾瓶罐外表液體,以防滑溜。使用後沾染到藥品處以洗滌瓶噴以蒸餾水沖洗。
\item \tb{取藥過量}:勿將過剩藥品倒回原瓶內。
\item \tb{移液器具}:一種溶液配用一組吸量管、滴管等移液器具,若需重複使用應先洗淨。轉移溶液到其他容器內時,需在液面上洩出,勿在將管尖浸入液面。
\item \tb{傾倒液體}:傾倒液體時,標籤應朝手心或朝上,以免溢流毀損標籤與造成下次取用時損傷手部,持一玻棒抵住杯嘴(如有)或杯緣,另一端靠在接收容器或漏斗的內壁上,玻棒應與內壁呈小角度,傾斜容器,使液體沿玻捧緩慢導入另一容器,傾出所需液量後,即將容器扶正但誤移開,等待最後一滴流下方可移除玻棒。
\item \tb{藥品濺於桌面}:以乾布擦去後以大量水沖洗,若為濃酸、濃鹼,分別以弱鹼(如氨水)、弱酸(如醋酸)中和再以水沖洗,或逕用大量水稀釋、沖洗。
\item \tb{配置硫酸水溶液}:應將濃硫酸緩慢沿玻棒加入水中,不可將水加入濃硫酸中,以免沸騰、飛濺。
\item \tb{取用鹼金屬}:用小刀切取,實驗時顆粒不宜太大,否則易造成危險,一般鈉在 5 mm $\times$ 5 mm $\times$ 5 mm 以下為佳、鉀在 3 mm $\times$ 3 mm $\times$ 3 mm 以下為佳。
\item \tb{研磨}:氯酸鉀等易因打擊而爆炸之藥品不可研磨。
\end{itemize}
\sssc{乾燥劑}
\bit
\item 通用乾燥劑:矽膠(silica gel)\ce{$m$SiO2$\cdot n$H2O(s)}、\ce{Mg(ClO4)2(s)}、\ce{CaCl2(s)}、\ce{CaSO4(s)}、分子篩(molcular sieve)、活性碳(Active/Activated carbon)/活性炭(Active/Activated charcoal)、沸石(Zeolite)(如鈉沸石(Natrolite)\ce{Na2Al2Si3O10$\cdot 2$H2O})。
\item 天然乾燥劑:玉米粉、纖維素。
\item 酸性乾燥劑(指不能用於乾燥鹼性氣體):\ce{P4O10(s)}、濃\ce{H2SO4(aq)}。
\item 鹼性乾燥劑(指不能用於乾燥酸性氣體):\ce{Ca(OH)2(s)}、\ce{CaO(s)}、\ce{NaOH(s)}。
\eit
\sssc{氣體}
\bit
\item \tb{氣體收集方法}:
\begin{itemize}
\item \tb{排水集氣法}:適用於難溶於水的氣體。將氣體排入在水中倒置的廣口瓶,欲取出時以錶玻璃等在水下蓋住瓶口再取出。其中會有水的飽和蒸氣壓的水蒸氣。對於不可溶於水的氣體,此法純度較向上或向下集氣法高,故一般溶解度難溶及以下之氣體均用此法,微溶之氣體欲較高純度亦用此法。
\item \tb{向下排/集氣法}:適用於密度比空氣(平均分子量 28.8)大的氣體。將氣體排入正置的廣口瓶,瓶口以錶玻璃等遮掩,僅流供氣體排入之縫隙。
\item \tb{向上排/集氣法}:適用於密度比空氣小的氣體。將氣體排入倒置的廣口瓶,瓶口以錶玻璃等遮掩,僅流供氣體排入之縫隙。
\item \tb{注射筒法}:將反應容器蓋以橡皮塞,將集氣管插入其中,另一端接上注射筒,緩慢拉動活塞集氣。
\end{itemize}
\item \tb{氣體吸收劑}:強酸溶液吸收鹼性氣體;強鹼溶液吸收酸性氣體。
\item \tb{抽氣櫥、通風櫥與抽氣罩}:若實驗會生成有毒或惡臭的氣體,如氯化氫、硫化氫、氯氣、二氧化氮、氨等,應於抽氣櫥或通風櫥內操作,若僅是一般無毒有機蒸氣等,可用局部抽氣的抽氣罩在容器之上抽氣。
\item \tb{搧聞法}:嗅聞試藥的氣味時,必須將臉與容器隔出一段距離,再用手輕輕搧動,揮引氣體而嗅聞之。
\item \tb{氣壓}:大量產生氣體之反應及抽氣減壓之裝置,應預防壓力過大使橡皮塞或軟木塞噴出或裝置破裂。
\eit
\sssc{人}
\bit
\item 頭髮過長應紮起,以免遭火燒或藥品沾染。
\item 穿實驗衣、戴護目鏡。
\item 實驗前熟悉實驗步驟。
\item 不進行實驗以外活動。
\item 實驗過程不可離去,即使靜置蒸發等亦須注意突發狀況。
\item 吸入氣體致身體不適,應至室外通風處仰臥並深呼吸,必要時並就醫。
\item 腐蝕性藥品濺於皮膚上,立刻用大量水沖洗後送醫。
\item 腐蝕性藥品濺入眼睛,立刻於洗眼臺用大量水沖洗,再用消毒過的紗布擦乾,滴上眼藥水,並迅速就醫。
\item 玻璃等碎片濺入眼睛,立刻於洗眼臺用水沖出,並迅速就醫。
\item 腐蝕性藥品濺於衣服上,脫去後用大量水清洗。
\item 加熱後器具冷卻前不可徒手觸摸,以免灼燒,可隔以乾布觸摸。
\eit
\sssc{廢棄物處理}
\begin{itemize}
\item 重金屬:倒入重金屬廢液筒回收。
\item 汞:盡量回收,不可接觸皮膚,餘者灑上硫粉,使生成硫化汞(II)以降低毒性,再集中處理。
\item 過錳酸根與二鉻酸根:
\bit
\item 以硫代硫酸鈉水溶液在酸性條件下處理:
\[\ce{8MnO4^-(aq) + 5S2O3^{2-}(aq) + 14H+(aq) -> 8Mn^{2+}(aq) + 10SO4^{2-}(aq) + 7H2O(l)}\]
\[\ce{4Cr2O7^{2-}(aq) + 3S2O3^{2-}(aq) + 26H+(aq) -> 8Cr^{3+}(aq) + 6SO4^{2-}(aq) + 13H2O(l)}\]
\item 以熱草酸鈉水溶液在酸性條件下處理:
\[\ce{2MnO4^-(aq) + 5C2O4^{2-}(aq) + 16H+(aq) -> 2Mn^{2+}(aq) + 10CO2(g) + 8H2O(l)}\]
\[\ce{Cr2O7^{2-}(aq) + 3C2O4^{2-}(aq) + 14H+(aq) -> 2Cr^{3+}(aq) + 6CO2(g) + 7H2O(l)}\]
\item 以焦亞硫酸鈉加入水中形成的硫酸氫鈉水溶液在酸性條件下處理:
\[\ce{Na2S2O5(s) + H2O(l) -> 2Na+(aq) + 2HSO3^-(aq)}\]
\[\ce{2MnO4^-(aq) + 5HSO3^-(aq) + H+(aq) -> 2Mn^{2+}(aq) + 5SO4^{2-}(aq) + 3H2O(l)}\]
\[\ce{Cr2O7^{2-}(aq) + 3HSO3^-(aq) + 5H+(aq) -> 2Cr^{3+}(aq) + 3SO4^{2-}(aq) + 4H2O(l)}\]
\item 以鐵(II)鹽水溶液在酸性條件下處理:
\[\ce{MnO4^-(aq) + 5Fe^{2+}(aq) + 8H+(aq) -> Mn^{2+}(aq) + 5Fe^{3+}(aq) + 4H2O(l)}\]
\[\ce{Cr2O7^{2-}(aq) + 6Fe^{2+}(aq) + 14H+(aq) -> 2Cr^{3+}(aq) + 6Fe^{3+}(aq) + 7H2O(l)}\]
\eit
再倒入重金屬廢液筒回收。
\item 鈉:放入酒精中以生成乙醇鈉與氫氣,直到不再有氣泡(氫氣)生成,再以酸中和,再倒入排水管。
\item 電石:緩緩加水,直到不再冒出氣泡為止,再將固體丟入垃圾筒、液體倒入排水管。
\item 有機溶劑:應回收至有機溶劑的廢液筒中,集中處理,惟甲醇與乙醇可逕倒入排水管。
\item 不含重金屬之酸鹼:先用水稀釋再用適當稀酸鹼中和之,再倒入排水管。
\item 玻璃碎片:小心全部清掃集中再回收。
\item 其他:投入特定的回收容器中。
\end{itemize}
\ssc{質量度量}
\sssc{電子天平}
使用步驟:
\ben
\item 保持清潔、乾燥、平穩與水平放置,避免震動。遺落之試劑務必立即以毛刷掃除。
\item 使用前將稱量紙或容器放在秤盤上,待顯示數值穩定後按「O/T」鍵歸零。
\item 加入待測物,待顯示數值穩定後即可得待測物之質量。
\een
\ssc{上皿天平/平臺天平}
使用步驟:
\ben
\item 保持清潔、乾燥、平穩與水平放置,避免震動。遺落之試劑務必立即以毛刷掃除。
\item 使用前將稱量紙或容器放在秤盤上,調整校準螺絲,校準螺絲向內旋入該端會變輕,反之則變重,使指針刻度為零或左右擺動的幅度相同歸零。
\item 砝碼應使用砝碼夾夾取,不要用手觸碰砝碼以免生鏽。物體與砝碼放置時,應靠近秤盤中央,砝碼應由大至小依序放置。
\item 指針朝上,偏向重端,應添加砝碼直到指針刻度為零或左右擺動的幅度相同,此時左盤質量等於右盤質量。
\een
\sssc{懸吊式等臂天平}
使用步驟:
\ben
\item 保持清潔、乾燥、平穩與水平放置,避免震動。遺落之試劑務必立即以毛刷掃除。
\item 使用前將稱量紙或容器放在秤盤上,騎碼移至零讀數,調整校準螺絲,校準螺絲向內旋入該端會變輕,反之則變重,使指針刻度為零或左右擺動的幅度相同歸零。
\item 物體在左盤,砝碼在右盤。砝碼應使用砝
碼夾夾取,不要用手觸碰砝碼以免生鏽。物體與砝碼放置時,應靠近秤盤中央,砝碼應由大至小依序放置,而後調整騎碼。
\item 橫梁上有騎碼,橫梁上標示讀數,騎碼位於之讀數應加於右盤之重量。
\item 指針朝下,偏向重端,應添加砝碼與調整騎碼直到指針刻度為零或左右擺動的幅度相同,此時左盤質量等於右盤質量加上騎碼讀數。
\een
\sssc{三梁天平}
使用步驟:
\ben
\item 保持清潔、乾燥、平穩與水平放置,避免震動。遺落之試劑務必立即以毛刷掃除。
\item 僅有一秤盤,使用前將稱量紙或容器放在秤盤上,騎碼移至零讀數,調整校準螺絲,校準螺絲向內旋入會變輕,反之則變重,使指針刻度為零或上下擺動的幅度相同歸零。
\item 三橫梁上各有質量不同之騎碼,各橫梁上標示讀數,通常分別是 0.01 g、0.1 g 和 1 g 為單位,三橫梁上騎碼位於之讀數各乘以其單位後加總,騎碼應由質量大至小依序調整。
\item 指針方向與桿平行並背向秤盤,物體重於騎碼時偏向上,反之偏向下,應調整騎碼直到指針刻度為零或上下擺動的幅度相同,此時物體質量等於三橫梁上騎碼讀數各乘以其單位後總和。
\een
\ssc{體積度量}
\sssc{體積度量}
\bit
\item 度量用儀器(如量筒、容量瓶)不可用於長期存放試劑,使用後應立即用清水沖洗乾淨,再用蒸餾水淋洗後,晾置滴乾。
\item 量取液體時,若液體為無色或淺色,則讀取液面中央處高度之刻度(水為中央低於四周、酒精為中央高於四周);若有顏色則讀取液面與管壁接觸之刻度;讀取時需以平視法。
\item 度量儀器依製造時校正體積的方式,可略分為兩類:
\bit
\item \tb{TC/To Content/In/Inclusive}:加液體到標線時,內含液體體積為容器上標註體積,常標為 In 或 TC。例如:量筒、容量瓶。
\item \tb{TD/To Deliver/Ex/Exclusive}:移轉放出的液體體積為標線刻度所指示的體積,常標為 Ex 或 TD。例如:滴定管、體積吸量管。
\eit
\item 度量用儀器不可加熱,其傳熱不均勻可能爆裂,並會使刻度不準確。
\item 液體體積測量之精確度:體積吸量管>分度吸量管$\approx$容量瓶>量筒>非度量用儀器(如燒杯、錐形瓶),體積吸量管、分度吸量管與容量瓶視為精準,其餘視為不精準。
\eit
\sssc{容量瓶/定量瓶/量瓶(Volumetric flask, measuring flask, or graduated flask)}
容量瓶是細頸梨型平底玻璃瓶,由無色或棕色玻璃製成,頸部刻有標線,精確度高。帶有磨砂玻璃塞或橡皮塞,合適的瓶塞可用繩繫在瓶頸上。

使用步驟:
\ben
\item 使用前先檢查瓶塞是否吻合。
\item 固體溶質應先在燒杯中完全溶解後,再利用玻棒或漏斗轉移到容量瓶,溶液倒盡後應再用少量溶劑潤洗燒杯等用具數次,潤洗液按相同方法移入容量瓶中。
\item 而後以溶劑加入容量瓶,當溶液達三分之二容量時,將容量瓶沿水平方向輕輕擺動使溶液初步混勻,最後以溶劑加到標線。
\item 蓋緊瓶塞,一手食指按住瓶塞,另外四指拿住瓶頸,用另一手托住瓶底,將瓶倒轉使氣泡上昇到頂,如此反覆十餘次可讓溶液混合均勻。
\item 溶液需調整至實驗溫度後才能倒入量瓶中,否則溶液體積會有誤差。
\een
\sssc{(移液)吸(量)管/移液管/移液器(Pipette or pipet)與安全吸球(Pipette filler)}
\begin{itemize}
\item \tb{體積(Volumetric)/球形(Bulb)/腹式(Belly)/移液(transfer)吸(量)管/移液管/移液器}:有一個大球狀或圓柱狀體,其上方有一個狹長部分,並帶有單一刻度標記,用於吸取固定體積的液體,精確度極高,可達四位有效數字。
\item \tb{分度(Graduated)吸(量)管}:管壁具有細刻度,可任取所需體積之液體,精確度高。
\item \tb{安全吸球(Pipette filler)}:橡膠製,一大球上方與下方分別以一短管連至控制處 A、S,A 上方連一開口短管通至外界空氣,S 下方有一小球,小球一側以一短管連至控制處 E,小球下方連一開口短管,用於將吸量管上端套入其中,E 旁連一開口短管通至外界空氣,各控制處內有小玻璃珠或鋼珠,按壓時能開啟管路讓空氣流通,其中 A 用於排氣,S 用於吸取溶液,E 用於釋放液體。
\eit

使用步驟:
\begin{enumerate}
\item 使用前先清洗乾淨至內壁無水珠附著,將下端內外的水拭去,並可用少量欲移取的溶液(約吸量管容積的五分之一)潤洗 2-3 次避免液體被殘留在管壁的水稀釋,潤洗過之溶液不可回收使用。
\item 小心插入安全吸球的下端,必要時可沾少許蒸餾水潤滑。姆指及食指按壓 A,以其它手指擠出大球中空氣後,再放鬆 A。
\item 將尖端浸入欲取用溶液中,用姆指及食指按壓 S,小心控制使溶液被吸取至所要的刻度處,即停止按壓 S 之動作,若吸取過多可按壓 E 釋放溶液。取樣過程中,應避免溶液被吸入安全吸球內,若不慎吸入,需將液體擠出並洗淨,否則易造成汙染,並易可能因腐蝕或沾黏導致吸球損壞。
\item 將吸量管連同吸球移離溶液,放至欲加入之容器內。
\item 按壓控制 E,使吸量管釋放所要體積的溶液。擠壓 E 旁之小球,可使殘留於管尖的殘留液流出。
\item 安全吸球用完務必要放氣使吸球恢復原狀,避免吸球因彈性疲乏而損壞。
\end{enumerate}
\sssc{滴定(Titration)與滴定管(Burette or buret)}
滴定是一種定量化學分析方法,通過滴加滴定液直到滴定終點,由所需的滴定液量確定待測液中已知化學物質的未知濃度的過程。

點:
\bit
\item \tb{當量點(Equivalent point)}:滴定試劑與待測試劑當量數相等之時點。
\item \tb{半當量點(Half-equivalence point)}:加入之滴定液體積為當量點時之一半之時點。
\item \tb{滴定終點(Titration end point)}:指變色或其他可測事件的發生,作為滴定的結束。
\eit

滴定管分類:
\bit
\item \tb{酸式滴定管}:使用玻璃珠控制流速,不耐鹼與氫氟酸。
\item \tb{鹼式滴定管}:使用硫化橡膠活栓控制流速,不耐氧化劑與酸。
\item \tb{酸鹼兩用滴定管}:使用聚四氟乙烯活栓控制流速,耐酸、耐鹼、耐氧化劑,可用於裝各種滴定液。
\eit

滴定步驟:
\begin{enumerate}
\item 檢查活栓或玻璃珠並調整之。
\item 洗淨後用少量(約滴定管容積的五分之一)滴定液潤洗 2-3 次避免液體被殘留在管壁的水稀釋,潤洗過之溶液不可回收使用。
\item 以滴定管夾將滴定管固定在支撐架上。
\item 以漏斗緩緩將滴定液加入滴定管中。
\item 將一燒杯錐形瓶置於支撐架底座上滴定管下方處,轉動活栓釋放少量液體以排除下端氣泡,記錄滴定管最初讀數。
\item 將待滴定之溶液與指示劑(如需)加入另一錐形瓶中,置於支撐架底座上滴定管下方處,可在下方置一白紙或白瓷板便於觀察溶液顏色變化。
\item 一手以左手拇指、食指及中指控制活栓,調整滴定液流速,一手拿住錐形瓶頸,沿同一方向按圓周搖動錐形瓶。
\item 開始滴定時無明顯變化,流速可快一些,若滴落點周圍開始出現暫時性的判斷滴定終點的性質(通常為顏色)的變化但隨溶液搖晃或攪拌很快消失,此時應逐滴加入,愈接近滴定終點,消失愈慢,至出現終點應有的性質 30 秒以上不改變時,即是滴定終點。
\item 記錄滴定管最後讀數,兩次讀數差便是使用滴定劑之容積數。使用完畢後,將滴定管內剩餘溶液棄去,並洗淨滴定管。
\end{enumerate}

\tb{標定}:若欲使用之滴定液當量濃度未知,須先以之滴定已知濃度的可與之反應的試劑,以推知滴定液當量濃度,稱標定。許多物質具潮解性、照光或加熱易分解或會與空氣反應,經久置其濃度及品質易改變,如氫氧化鈉水溶液、過錳酸鉀水溶液,須先標定。
\sssc{量筒(Graduated cylinder or measuring cylinder)}
窄圓柱形透明容器,每條刻度線代表已測量的液體體積。
\ssc{加熱}
\sssc{加熱}
\bit
\item 試管以外容器不可直接加熱,以免破裂。
\item 度量器具不可加熱亦不可以烘箱乾燥。
\item \tb{突沸/暴沸(Boiling chip)}:指不穩定的過熱液體突然劇烈沸騰的現象。加熱試管以外容器時應攪拌,以免與熱源接觸的部分過熱產生突沸,並可在開始加熱前以漏斗於溶液中加入 2-3 個沸石或毛細管,加熱時可吸附受熱產生的氣體,使氣泡結構受到破壞,產生小氣泡,小氣泡上升可達到對流效果,進而防止突沸。不可在加熱過程中加入新物質,因為液體溫度若已達沸點,加入新物質可能會引起突沸。沸石與毛細管不能重複使用。
\item \tb{滅火}:若酒精燈不慎翻倒造成小面積著火,可以使用溼抹布或滅火毯蓋滅,也可用水澆熄,因乙醇可溶於水中。普通火勢可用溼抹布、滅火砂、滅火毯、滅火器等撲滅。火熄滅後,不可立即移開遮蓋物,以免復燃。氧化電位大於碳之金屬可在二氧化碳中燃燒,故不可用二氧化碳滅火器滅火。滅火後應打開門窗通風,使蒸氣排散。
\item \tb{試管直接加熱}:加熱時,以試管夾斜持於熱源上,試管口不可朝人,以免管內物質濺出傷人,液體不應超過三分之一,加熱會產生水氣或液體的固體試藥或反應,應將試管口略微朝下,以免凝結的水滴倒流至試管底部造成破裂。
\item \tb{隔水/水浴(Water bath)與隔油/油浴(Oil bath)加熱}:將試藥置於試管中,再將試管置於裝有水或其他液體的燒杯中,接著將燒杯以陶瓷纖維網與三角架間接加熱。或將試藥置於燒杯等中,再將燒杯等置於水浴器中加熱。此法溫度上升較均勻也較緩慢,適用於揮發性溶劑或易燃試藥,如有機液體。
\eit
\sssc{水浴器}
一水槽內有熱水,下有熱源,燒杯等可置於其中加熱,通常附有比例積分微分控制器(Proportional–integral–derivative controller, PID controller)以恆溫,須注意不使水浴器之水進入容器中或使容器中之物質漏至水浴器中。
\sssc{加熱板(Hot plate)}
加熱板提供變動的磁場,以電流熱效應加熱容器,可用磁性攪拌子(Magnetic stirrers/mixers/stir bars)攪拌,其可在使用加熱板時自行旋轉。
\sssc{加熱包}
包覆著絕緣物的電阻加熱器,一般可達到攝氏450度,可能需要配合變壓器使用。
\sssc{蒸汽浴(Steam bath)}
帶有一系列可拆卸的同心環,以適應不同燒杯、錐形瓶和圓底燒瓶的大小,適用於揮發性溶劑或易燃試藥,如有機液體。
\sssc{酒精燈(Alcohol burner)}
使用步驟:
\begin{enumerate}
\item 新酒精燈需要配置燈芯,將燈芯修剪到浸入酒精 4 至 5 cm,可以調整露出陶瓷套管的燈芯長度,來控制火焰的高度,但不宜超過 3 mm,否則會使火焰不穩定。
\item 對於舊燈,尤其是長時間沒有使用的酒精燈,應該取下燈帽並提起陶瓷套管,用洗耳球或嘴輕輕地向燈壺內吹氣,以趕走其中聚集的酒精蒸氣,接著再檢查燈芯,如果燈芯不整齊或者燒焦的話,應用剪刀修剪整齊。
\item 使用前還應該檢查燈壺是否有破損,並檢查酒精儲量,酒精若少於容積的二分之一,就應該添加酒精,但酒精不能超過容積的三分之二。添加酒精應將陶瓷套管取出,用漏斗或者玻璃棒引流添加,以免灑出,點燃的酒精燈應先熄滅再添加,否則容易引起火災。添加酒精後應擦拭桌面與壺外壁使保持清潔、乾燥,以免點火時發生危險。
\item 點燃酒精燈前要調整燈芯,使得燈芯浸滿酒精,否則容易將燈芯燒焦。點燃酒精燈必須使用火柴或打火機,不能用已經點燃的酒精燈來引燃另一個酒精燈,否則容易使酒精灑出引起火災。
\item 使用中的酒精燈,應放置在適當的高度,使用木塊墊高,並可以使用專用擋板擋風,均不能使用書本。使用中不可以傾斜拿取和放置,並謹防碰倒酒精燈。
\item 如果需要穩定火焰以及提高火焰溫度,可以在火焰周圍增加一個金屬網罩。
\end{enumerate}
\sssc{本生燈(Bunsen burner)}
燃燒煤氣、液化石油氣或天然氣(下通稱瓦斯),溫度較酒精燈高,現實驗室除非需要更高溫度否則多用酒精燈。

構造:
\bit
\item \tb{燈管}:長管。
\item \tb{出氣口/燈管口}:燈管上端火焰所在之處,位在本生燈最上方。
\item \tb{氣孔}:供空氣進入提供氧氣之處,位於燈管之下、 膠喉之上。
\item \tb{膠喉}:連接瓦斯管,位在本生燈最下方。
\item \tb{空氣調節螺栓}:位在氣孔之上,控制氣孔大小以控制空氣進氣流量,逆時針旋動增加、順時針旋動減少。
\item \tb{瓦斯調節螺栓}:位在氣孔之下,控制瓦斯進氣流量,逆時針旋動增加、順時針旋動減少。
\item \tb{瓦斯開關活栓/煤氣掣}:位在膠喉上,控制瓦斯管上的瓦斯流量,活栓與瓦斯管夾角愈大,流量愈大,平行時流量最大,垂直時完全關閉。
\eit

使用步驟:
\begin{enumerate}
\item 開窗保持空氣流通。
\item 將本生燈放在防火板上。
\item 將膠喉與瓦斯管接上。
\item 關閉瓦斯調節螺栓、空氣調節螺栓與瓦斯開關活栓。
\item 將點燃的火柴或點火槍的電火花噴嘴放在出氣口上。
\item 開啟煤氣掣,把燈點燃,這時火焰通常為橙色。
\item 打開空氣調節器讓新鮮空氣進入與瓦斯混合,火焰逐漸轉為藍色且溫度變高,待火焰上端呈淡藍色且穩定不跳動,此時火焰分層:
\bit
\item \tb{外焰}:淡藍色,與外部無明顯界線,瓦斯完全燃燒,溫度極高,約在 1540-1570°C。
\item \tb{內焰}:亮藍色,圓錐狀,溫度稍低於外焰。
\item \tb{焰心}:藍黑色,靠近燈管開口處,瓦斯尚未完全燃燒,溫度僅約 350°C。
\eit
\item 開始加熱,應調整受熱器材高度使其底部在內焰圓錐尖端處,使充分以外焰加熱。
\item 使用中各情況處理方法:
\bit
\item 火焰呈橙黃色:表示空氣供給量不足,瓦斯燃燒不完全,必須逆時針旋動空氣調節螺栓增加進空氣量。
\item 火焰跳動、有顯著嘶吼聲,或騰空於燈管上方:表示空氣供給量過大,必須順時針旋動空氣調節螺栓減少進空氣量。
\item 火焰在燈管內部燃燒,或欲暫停使用 3 分鐘以上:關閉瓦斯調節螺栓與瓦斯開關活栓,待本生燈冷卻後再重新點火。
\item 點火不成功或火焰熄滅:立即關閉瓦斯調節螺栓與瓦斯開關活栓,接著關閉空氣調節螺栓與處理其他實驗細節,待瓦斯散去再重新點火。
\item 聞到瓦斯味:立即關閉實驗室內所有熱源,接著檢查漏氣來源。
\eit
\item 使用完畢後,先關閉空氣調節器螺栓,然後關閉瓦斯調節螺栓與瓦斯開關活栓,最後把膠喉從瓦斯管拔除。
\end{enumerate}
\sssc{坩堝(Crucible)}
上大下小的杯狀器皿,部分有蓋,用於熔融或灼燒試劑的耐火容器,通常是氧化鎂製成。

使用步驟:
\ben
\item 全程用乾淨的\tb{坩堝鉗(Crucible tong)}夾取。
\item 使用前應嚴格乾燥後再秤量。
\item 可用無灰濾紙過濾並將濾紙一起放進坩堝,這種濾紙在高溫環境下完全分解,不會影響結果。
\item 高溫處理後,將坩堝和所容物乾燥冷卻後再稱量。
\een
\sssc{三角架與陶瓷纖維網(Tripod with mesh screen)}
三角架下放置酒精燈,上放置陶瓷纖維網,上放置燒杯等欲加熱物,稱間接加熱。隔一片陶瓷纖維網可以使熱能均勻分散,避免容器破裂。
\sssc{烘箱(Laboratory oven)}
乾燥時應於烘箱內乾燥,不可用明火乾燥。
\sssc{燃燒匙(Burning spoon)}
長柄耐火小勺子,用於盛裝少量的藥品進行燃燒實驗,可伸入廣口瓶。
\sssc{泥三角(Pipeclay triangle)}
三根套有耐火黏土陶瓷管的金屬排成正三角形,用於支撐坩堝等物於本生燈上加熱。
\sssc{蒸發皿(Evaporating dish)}
淺盤至半球型耐熱容器,用於蒸發其中物質。
\ssc{壓力度量}
\sssc{開口式壓力計}
一 J 型管,矮側接一球,球內充以待測氣體後關閉閥門,U 型底部充滿密度$\rho$、蒸氣壓$p^{\circ}$液體(通常是水銀,其密度 13.59 g/cm$^3$,室溫下飽和蒸氣壓 0.0012 mmHg,故常忽略$p^{\circ}$),高側開口與外界壓力$p$大氣相通,高側與矮側液體高度分別為$H$、$h$,重力加速度$g$,此時待測氣體的壓力為:
\[p-p^{\circ}+(H-h)\rho g\]
\sssc{閉口式壓力計}
一 J 型管,矮側接一球,球內充以待測氣體後關閉閥門,U 型底部充滿密度$\rho$液體(通常是水銀),高側閉口,高側與矮側液體高度分別為$H$、$h$,重力加速度$g$,此時待測氣體的壓力為:
\[(H-h)\rho g\]
\sssc{托里切利實驗(Torricelli's experiment)}
托里切利(Evangelista Torricelli)發明,一充滿水銀開放水槽,其中倒立充滿水銀且內無氣體的足夠長一端密封、一端開口玻璃管,開口在水槽水銀液面下,則在外界大氣壓力一大氣壓時管內水銀液柱高 76 cm,外界大氣壓力 $x$ cmHg 時管內水銀液柱高 $x$ cm,管上端無水銀液體處僅有水銀飽和蒸氣壓(室溫下 0.0012 mmHg)的水銀蒸氣,稱托里切利真空(Torricellian vacuum),常近似為真空。此裝置還可用於測量密度小於汞且飽和蒸氣壓小於當下氣壓的液體的飽和蒸氣壓,用注射筒往管內注射待測液體,直到水銀液面上方出現薄薄的一層待測液體,此時若水銀蒸氣不計則管上端無液體處即為該液體的飽和蒸氣壓。
\ssc{溫度度量}
\sssc{溫度計(Thermometer)}
\bit
\item 測量液體溫度時,溫度計感溫點(通常是一充滿液體之小球)應於液體中心處,且不可碰觸容器壁。
\item 讀取刻度方法同體積度量容器。
\item 不可用溫度計攪拌溶液,以免破裂。
\eit
\sssc{卡計/熱量計/量熱計(Calorimeter)}
卡計的構造為,絕熱盒中一不鏽鋼杯,上有杯蓋,蓋上有一帶一孔之橡皮塞,用於插溫度計,杯中有一不鏽鋼製由長柄與圓圈構成的攪拌棒,圓圈面與長柄垂直,圓圈在杯中,長柄上端通過蓋上另一孔伸至外界,手握之以攪拌。宜用於測量較快速且反應熱絕對值較大之反應的反應熱,如酸鹼中和、固體溶解。

使用步驟:
\ben
\item 洗淨、擦乾卡計。
\item 備妥各反應物並測量溫度。
\item 備妥插有溫度計與攪拌棒的卡計杯蓋。
\item 將部分互不反應之反應物種倒入杯中。
\item 將剩餘反應物種迅速倒入杯中,立即蓋上杯蓋並插入溫度計與攪拌棒。
\item 每隔數秒記錄一次溫度,直到讀數連續 30 秒內溫度變化小於 0.1 K。放熱反應以溫度最大值為終止溫度、吸熱反應以溫度最小值為終止溫度。
\item 洗淨、擦乾卡計。
\een
\ssc{根據比重的分離過程}
\subsubsection{沉降(Sedimentation)}
沉降是固體向液體底部移動並形成沉澱的過程。
\subsubsection{傾析(Decantation)}
當固體顆粒的比重較大,靜置或離心能沉降至容器底部時,通過傾倒上層液體來分離沉澱物的方法。若固體顆粒細小、易隨液體流動或懸浮中,則需採用過濾法。
\subsubsection{離心(Centrifugation)}
離心是利用高速旋轉產生的離心力,增加離心​管的等效重力,使混合物中的成分按密度差異分層,固形物快速移動到管的底部,上方的澄清液體稱為上清液(Supernatant)。當固體顆粒量少但須回收,或是固體顆粒非常細小時,可使用離心分離。

使用步驟:
\ben
\item 勿使用一般試管代替離心管,一般試管管壁較薄,且尺寸形狀不能和離心機完全吻合,離心時易破裂。
\item 旋轉時若離心力不均會造成危險與機器損壞,故離心管需對稱放置,即兩離心管在離心機轉軸兩側相同距離處,且質量相等,若僅有一試樣亦需對稱擺放一裝有等重的水的離心管。
\item 放置好離心管並把離心機蓋子蓋妥後方能啟動。離心機仍在旋轉時不可打開蓋子。停止時應讓離心機自行停止,絕不可用手或器具強制它停止轉動。
\een
\subsubsection{淘析(Elutriation)}
主要用於小於 1 μm 的顆粒,利用與沉降方向相反的方向流動的氣體或液體流,使最終沉降速度低於上升流體的速度的較輕顆粒上升到頂部並溢出。
\ssc{根據顆粒大小的分離過程}
\subsubsection{重力過濾(Gravity filtration)/常壓過濾/簡單過濾}
利用濾材/過濾介質(如篩網、濾紙或濾膜)截留混合物中的固體顆粒,過濾的速率會受到溫度、黏度、固體顆粒和濾材孔隙的大小、其他性質等影響。
\begin{itemize}
\item \tb{濾紙(Filter paper)}:應按照正確的方式折疊,圓形濾紙折疊普通折法為對折再對折,並撕下一角,折好後應用少量溶劑或純水溼潤,以更緊貼漏斗內壁。
\item \tb{過濾漏斗(Filter funnel)}:有毛細管,可以提高過濾效率。
\item \tb{濾紙支架}:使用過濾漏斗應搭配濾紙支架,以防止濾紙滑動。
\end{itemize}
\subsubsection{抽氣過濾/抽濾/減壓過濾(Suction filtration or vacuum filtration)}
利用抽氣裝置產生的負壓加速過濾,抽濾的速度比簡單過濾快,過濾物也比簡單過濾乾燥。
\bit
\item \textbf{布氏漏斗(Büchner funnel)}:傳統上由白色瓷製成的。漏斗形部分的頂部有一個圓柱體,圓柱體上有一個燒結玻璃盤或多孔板,將其與漏斗分開。帶有燒結玻璃盤的漏斗可以直接用於過濾;帶有多孔板的漏斗,需將濾紙形式的過濾材料放置在板上,濾紙需選用或裁切為能遮蓋漏斗圓形平板上的所有小孔,但不可大於圓形平板,否則濾紙無法平貼,會造成減壓時漏氣、過濾速度緩慢、固體漏至濾液。開始抽氣前濾紙可先潤溼以利平貼於漏斗。
\item \tb{布氏燒瓶(Büchner flask or Bunsen flask)/抽濾瓶(Vacuum flask, filter flask, suction flask, or side-arm flask)}:側邊有短管的錐形瓶,短管處用於以厚壁的橡皮管和抽氣裝置連接。
\item \tb{水流抽氣幫浦}:水通過抽氣幫浦,將布氏燒瓶中的氣體吸出,使其外部和內部之間存在壓力差,使布氏漏斗的內容物被吸向布氏燒瓶內。應確保設備密封,防止空氣通過,並避免物理應力點破壞器具。過濾完成後,先拆卸橡皮管再關閉抽氣裝置,否則使用水流抽氣幫浦時可能導致幫浦內的水被抽至仍處於低壓的布氏燒瓶。
\eit
\ssc{色譜法/色譜分析法/層析法/色層分析法(Chromatography)}
層析利用不同物質對流動相與固定相的附著力不同,以流動相對固定相中的混合物進行洗脫,不同組分沿固定相移動的速率不同,最終達到分離的效果。
\subsubsection{固定相/(層析)靜相(Stationary phase)}
層析柱中不動的部分,通常是固體或黏附在固體支撐上的液體。
\sssc{流動相/洗脫相/移動相(Mobile phase)}
與樣品混合並通過固定相的部分,通常是氣體或液體。流動相的作用是攜帶樣品通過固定相,幫助樣品中的組分移動和分離。
\subsubsection{$R_f$ 值/保留因子(Retention factor)}
指樣品爬升高度與流動相爬升高度的比值,其值愈大,樣品對移動相的吸附力相對於對固定相的吸附力愈大。
\subsubsection{薄層層析法(Thin-layer chromatography, TLC)}
以薄層色譜片/TLC 片/層析板/層析片(TLC plate)為固定相進行層析。

構造:
\begin{itemize}
\item \textbf{支撐底物}:通常是一個堅固的板或片,常見的材料有玻璃、鋁箔或塑膠。
\item \textbf{固定相層}:塗覆在支撐底物上,常見的固定相包括:
\begin{itemize}
\item 矽膠(Silica gel)或鋁土(Alumina):用於極性分離。
\item 纖維素:用於分析較大分子的分離。
\end{itemize}
\item \textbf{黏合劑}:在某些情況下,固定相層會與支撐底物透過黏合劑結合。
\end{itemize}

使用步驟:
\begin{enumerate}
\item 可先以鉛筆(不可用墨水筆)繪製起始線和停止線,前者距離 TLC 片底部約1-1.5公分。
\item \tb{點樣}:將試樣溶液用毛細管在起始線上點若干下(次數根據樣品濃度而定),每次在前一次點的溶液乾燥後才能再點,試樣耐熱者可吹乾,不可連續點,點面積愈小分離效果愈好。
\item \tb{蒸發}:靜置或加熱以使溶劑完全蒸發。若溶劑難以揮發,可在點樣後將板放於真空容器中乾燥後再使用。若溶劑未蒸發,殘留的溶劑會與流動相/展開液作用,降低流動相的均一性,導致分離效果變差。
\item \tb{展缸}:將少量流動相(對於有機物常用乙醇、異丙醇、丙酮等)倒於展缸/展開槽中,讓流動相高度小於起始線與層析片底部距離,蓋上錶玻璃並密封,使流動相蒸氣在展缸中飽和。可在展缸底部放上一張濾紙,讓濾紙底部浸沒於流動相中並靠在展缸內壁,以幫助流動相蒸發。
\item \tb{層析}:將 TLC 片置於展缸內,樣品點不可觸碰流動相,蓋上錶玻璃並密封,讓流動相通過毛細現象緩慢爬升。流動相遇到樣品混合點時,會帶著樣品上升(即洗脫樣品)。當流動相液前到達停止線時,將之拿出,迅速記錄流動相與樣品到達的高度並晾乾。不要讓溶劑爬升到層析板的頂部。
\item \tb{顯色}(如需):
\bit
\item 蛋白質:噴茚三酮(Ninhydrin)顯色劑;
\item 烷和鹵烷以外的有機物:將碘晶體置於廣口瓶,放入 TLC 片,加蓋,待產生深藍、紫紅至暗棕錯合物斑點後取出;
\item 螢光物質:照紫外燈。
\eit
\end{enumerate}
\subsubsection{紙層析法(Paper chromatography)}
與薄層層析法相似,惟以紙代替 TLC 片。
\subsubsection{管柱層析法(Column chromatography)}
將固定相(通常為固體顆粒)填充在管柱中進行層析。
\subsubsection{高效液相層析法(High-performance liquid chromatography, HPLC)}
利用高壓液體作為流動相進行層析。
\subsubsection{離子層析法(Ion chromatography/Ion-exchange chromatography)}
利用離子交換樹脂作為固定相進行層析。
\sssc{氣相層析法(Gas chromatography, GC)}
利用氣體作為流動相進行層析。
\ssc{萃取(Extraction)}
萃取是利用系統中組分在不同溶劑中有不同的溶解度來分離混合物的過程。
\subsubsection{浸出(Leaching)/固-液萃取(Solid–liquid extraction)}
利用溶劑將固體混合物中的一種或多種成分溶解出來的過程。
\subsubsection{液-液萃取(Liquid–liquid extraction)}
液-液萃取是利用兩種互不相溶的液體,將溶質從一相轉移到另一相的過程。例如用有機溶劑從水中萃取碘。

\tb{分液漏斗(Separatory funnel)}:頂部有塞(一般為玻璃栓塞),底部有閥門,分為梨形(圓錐形)與圓柱形等,梨形最常用。

使用步驟:
\ben
\item 取一分液漏斗,關閉底部的閥門,放在漏斗架上,開口處插上一般漏斗。
\item 從開口處之漏斗依序添加兩相。
\item 關閉頂部栓塞並按住之,多次緩慢翻轉漏斗與輕輕搖晃,使兩相混合。如果兩種溶液混合得太劇烈,會形成乳液與氣泡,應避免之。
\item 將分液漏斗倒置並小心地打開閥門以釋放多餘的蒸氣壓。
\item 將分液漏斗靜置於漏斗架上直到各相完全分離。
\item 打開頂部栓塞和底部閥門,透過重力釋放下部相。打開頂部栓塞是為了使漏斗內部和外界大氣之間壓力平衡。
\item 下部相被移除後,關閉閥門,將上部相透過頂部倒入另一個容器中。
\een
\subsubsection{超臨界流體萃取(Supercritical fluid extraction, SFE)}
超臨界流體萃取是利用超臨界流體(如超臨界二氧化碳)作為溶劑,從固體或液體中萃取特定成分的過程。其優點為超臨界流體在常溫常壓下易揮發為氣體,故毋須另外分離溶劑。例如以超臨界二氧化碳萃取咖啡因。
\subsubsection{固相萃取(Solid-phase extraction, SPE)}
即層析法。
\ssc{利用外加場的分離過程}
\sssc{場流分離(Field flow fractionation)}
利用外加場(如電場、磁場或等效重力場)對液體樣品中不同組分根據其與不同場的不同相互作用程度進行分離的過程。
\sssc{電泳(Electrophoresis)}
電泳是利用電場作用使帶電粒子在介質中移動,從而實現分離的技術,常用於生物大分子。
\ssc{根據沸點差異的分離過程}
\subsubsection{乾燥(Drying)/蒸發(Evaporation)}
通過蒸發除去樣品中的液態溶劑的過程。
\subsubsection{蒸餾(Distillation)}
蒸餾是利用液體混合物中各成分的沸點差異,通過加熱蒸發和冷凝分離各成分的過程。通常適用於沸點相差大於攝氏30度且不形成共沸物(Azeotrope)的兩種物質的混合物。

\bit
\item \tb{圓底燒瓶}:進行蒸餾時,需要選用容積約為試様溶液體積兩倍大小的圓底蒸餾瓶。若蒸餾瓶過小,所裝入的試樣溶液超過蒸餾瓶容量的\(\frac{2}{3}\),則沸騰時容易溢入收集瓶中。若是蒸餾瓶過大,所裝入的試樣溶液低於蒸餾瓶容量的\(\frac{1}{3}\),則殘留在蒸餾瓶中所損失的物質比率較多。
\item \tb{三叉管(Three prong clamp)}:用來銜接圓底燒瓶、溫度計及冷凝管的一種連接管。
\item \tb{溫度計}:插在圓底燒瓶頂的橡皮塞中,使感溫點的上端位置與三叉管的側管開口下端在同一水平面上,以正確測定蒸氣的溫度。若溫度計位置過高,所量得的溫度會偏低;若溫度計位置過低,所量得的溫度會偏高。
\item \tb{直形冷凝管(Straight condenser)/李必氏冷凝管(Liebigkühler, Liebig condenser)}:雙層玻璃套管,外管讓冷卻水(cooling water)通過,內管則讓蒸氣(steam)通過。使用時內管應乾燥無異物,冷卻水自外管下端流入,水壓不需太大,將內管的蒸氣冷卻凝結成液體,變成溫水由外管上端流出。若蒸氣的沸點高於100°C或在常溫下為固體,可用空氣冷卻而不通入冷水。連接冷卻水源與三叉管的橡皮管需套接緊密且夠深以免掉落,可先接好後再與三叉管連接。
\item \tb{連接彎管}:用於連接冷凝管與收集瓶的器具,上有一開口,以平衡蒸餾系統的內外氣壓,不可封住以免爆炸。減壓蒸餾時,可將抽氣減壓的橡皮管連接於此開口,逐漸降低蒸餾系統的壓力,在低壓下緩慢加熱,讓蒸物在較低溫度下沸騰。
\item \tb{收集瓶}:收集冷凝得到的蒸餾液。應事先洗淨、烘乾、稱重,以計算產率。不應使用廣口容器,以免蒸餾液揮發,一般使用錐形瓶。
\item \tb{磨砂口或橡皮塞}:整個蒸餾裝置的各個接頭可以使用磨砂口或橡皮塞相連接,前者氣密性更高。磨砂口可塗抹凡士林以潤滑與確保氣密性,塗抹後輕輕旋轉組件使均勻塗布,不可塗太多,以免隨蒸氣餾出。常壓蒸餾時可以不塗抹凡士林,只要確認接合緊密即可;減壓蒸餾則需塗抹凡士林。
\item \tb{加熱}:可使用水浴、油浴、蒸汽浴、加熱板、加熱包、間接加熱等,依欲餾物選擇。在開始加熱前以漏斗於溶液中加入 2-3 個沸石或毛細管。加熱溫度以使餾出液以約每秒 1 滴的速度滴出為適宜。蒸餾完成後,須立即移除熱源;蒸餾瓶中殘餘液不可完全蒸乾。先關熱源,後關冷卻水流。
\end{itemize}
\subsubsection{分餾(Fractional distillation)}
分餾是蒸餾的一種改進方法,通過分餾塔多次蒸發和冷凝,使混合物中各成分按沸點差異逐步分離。如煉油時加熱原油以分離出汽油、柴油、瓦斯等。
\ssc{使溶質沉澱的分離過程}
\subsubsection{再/重結晶(Recrystallization)}
再結晶是通過控制溶劑和溶質的溫度和濃度,從而獲得更高純度晶體的過程。例如,在溫度\(A\)時溶質甲全部溶解,溶質乙不完全溶解,則可將一溶有溶質甲和乙的溶液調整至溫度\(A\),取得溶質乙結晶。
\sssc{絮凝(Flocculation)}
絮凝是利用絮凝劑將細小膠體顆粒聚集成大顆粒,使其容易沉降或過濾的過程。
\ssc{利用選擇性通透的分離過程}
\subsubsection{逆滲透(Reverse osmosis, RO)}
利用半透膜在壓力作用下,將水中的溶質截留,使水分子通過膜的過程。
\subsubsection{透析(Dialysis)}
利用半透膜將小分子或離子從大分子或顆粒中分離出來的過程,常用於醫學與生物化學。
\subsection{CHNO 元素分析(CHNO elemental analysis)}
\begin{enumerate}
\item 封閉玻璃管中,置裝有已知質量之僅含 C、H、N、O 的待測物純質樣品的鉑小盤,其旁置已知質量之過量氧化銅(II),兩者之下各置加熱器。
\item 將已知質量過量純氧通入玻璃管,首先通過樣品而後通過作為氧化劑的氧化銅(II),兩者之下加熱器均加熱。樣品被氧化產生水蒸氣、二氧化碳和氮氧化物氣體。測量氧化銅(II)質量變化。
\item 待樣品完全反應,停止通入氧氣,使氣體通過裝有已知質量的過量顆粒固體狀過氯酸鎂或氯化鈣或十氧化四磷的封閉 U 型管(下稱此管為吸水管)。因其吸收水,測量其質量改變以得到樣品中氫的質量。
\item 接著使剩餘氣體通過過量銅,下置加熱器加熱,使銅將氮氧化物還原成氮氣。測量銅質量變化。
\item 接著使剩餘氣體通過裝有已知質量的過量顆粒固體狀氫氧化鈉或氫氧化鉀的封閉 U 型管。因其吸收二氧化碳,測量其質量改變以得到樣品中碳之質量。因其也會吸收水,故不可在通過吸水管前先通過之。
\item 測量剩餘氣體的密度以得到氮氣與氧氣的比。
\item 由質量守恆定律得到剩餘氣體的質量,再乘以氮氣的占比得到樣品中氮的質量,最後將樣品質量減去已知的氫、碳、氮質量得到氧的質量。
\end{enumerate}
\ssc{光譜學(Spectroscopy)}
\subsubsection{分光儀(Spectroscope or spectrograph)/光學光譜儀(Optical spectrometer)}
1821年,弗朗和斐利用稜鏡、狹縫繞射光柵與凸透鏡組合,發明第一臺分光儀。
\subsubsection{氣體放電發射光譜}
由某元素約 \scinote{6}{-2}至\scinote{2.6}{-5} atm 的低壓氣體經約 \tenpow{4} V 電壓的高壓放電管放出之光經光譜儀可得明線光譜。
\subsubsection{焰色測試(Flame test)發射光譜}
當加熱到高溫時,電子受到激發,回到基態時釋放特定波長的光,形成焰色,經光譜儀可得明線光譜。
\subsubsection{氣體吸收光譜}
連續光譜通過某元素之低溫氣體後,經光譜儀可得暗線光譜。
\subsubsection{紅外線光譜儀(Infrared spectroscopy, IR)}
利用化合物所含官能基吸收特定波長紅外線的分子光譜獲得訊息。
\subsubsection{核磁共振儀(Nuclear magnetic resonance, NMR)}
利用原子核在高磁場下躍遷至高能量狀態後回到較低能量時釋放的核磁共振訊號獲得訊息。
\ssc{反應進度的觀察方法}
\begin{longtable}[c]{|p{0.1\tw}|p{0.2\tw}|p{0.2\tw}|p{0.3\tw}|}
\hline
項目 & 測量 & 說明 & 實例\\\hline\endhead
顏色 & 利用分光光度計測量光度作為波長的函數 & 只適用於一當量反應物與一當量生成物具有不同顏色者 & 
\begin{itemize}
\item \ce{N2O4(g)\text{(透明)} <=> NO2(g)\text{(紅棕)}}
\item \ce{2N2O5(g)\text{(透明)} <=> 4NO2(g)\text{(紅棕)} + O2(g)\text{(透明)}}
\item \ce{H2(g)\text{(透明)} + I2(g)\text{(紫)} -> 2HI(g)\text{(透明)}}
\item \ce{2CrO4^{2-}(aq)\text{(黃)} + 2H+(aq)\text{(透明)} <=> Cr2O7^{2-}(aq)\text{(橘)} + H2O(l)\text{(透明)}}
\end{itemize}
\\\hline
透光度 & 比色法 & 只適用於一當量反應物與一當量生成物透光度不同者;基於比爾-朗伯定律(Beer–Lambert law) &
\begin{itemize}
\item \ce{2CrO4^{2-}(aq)\text{(黃)} + 2H+(aq)\text{(透明)} <=> Cr2O7^{2-}(aq)\text{(橘)} + H2O(l)\text{(透明)}}
\item \ce{Co(H2O)6^{2+}(aq)\text{(粉紅)} + 4Cl^-(aq)\text{(透明)} <=> CoCl4^{2-}(aq)\text{(藍)} + 6H2O(l)\text{(透明)}}
\item \ce{Fe^{3+}\text{(淡黃)} + SCN^-\text{(透明)} <=> FeSCN^{2+}\text{(血紅)}}
\end{itemize}
\\\hline
氣體壓力與體積 & 測量定壓體積變化率或定容壓力變化率 & 只適用於氣體反應物係數和不等於氣體生成物係數和者;基於氣體化合體積定律(Law of gas combining volume)、波以耳定律(Boyle's law) & 
\begin{itemize}
\item \ce{2NaHCO3(s) <=> NaCO3(s) + CO2(g) + H2O(l)}
\item \ce{N2(g) + 3H2(g) <=> 2NH3(g)}
\item \ce{N2O4(g) <=> 2NO2(g)}
\item \ce{2N2O5(g) <=> 4NO2(g) + O2(g)}
\item \ce{Zn(s) + 2CH3COOH(aq) -> Zn(CH3COO)2(aq) + H2(g)}
\item \ce{2H2O2(aq) -> 2H2O(l) + O2(g)}
\end{itemize}
\\\hline
釋放或吸收氣體 & 測量凝相重量 & 只適用於氣體反應物係數和不等於氣體生成物係數和者 & 
\begin{itemize}
\item \ce{2NaHCO3(s) <=> NaCO3(s) + CO2(g) + H2O(l)}
\item \ce{Zn(s) + 2CH3COOH(aq) -> Zn(CH3COO)2(aq) + H2(g)}
\item \ce{2H2O2(aq) -> 2H2O(l) + O2(g)}
\end{itemize}
\\\hline
沉澱 & 測量沉澱物重量 & 只適用於固體反應物係數乘以莫耳重量和不等於固體生成物係數乘以莫耳重量和者 & 
\begin{itemize}
\item \ce{Pb^{2+}(aq) + 2I^-(aq) -> PbI2(s)\text{(黃)}}
\item \ce{S2O3^{2-}(aq) + 2H+(aq) -> H2O(l) + SO2(g) + S(s)\text{(黃)}}
\item \ce{H2SO4(aq) + Ba(OH)2(aq) -> BaSO4(s)\text{(白)} + 2H2O(l)}
\item \ce{4Fe^{3+}(aq) + 3[Fe(CN)6]^{4-}(aq) -> Fe4[Fe(CN)6]3(s)\text{(藍)}}
\end{itemize}
\\\hline
pH/pOH 值 & 利用 pH 測計或酸鹼指示劑測量 & 只適用於一當量反應物與一當量生成物有不同質子濃度時 & 
\begin{itemize}
\item \ce{C2H5OH(aq) + HI(aq)\text{(酸)} <=> C2H5I(aq) + H2O(l)}
\item \ce{CH3COOH(aq)\text{(酸)} + C2H5OH(aq) <=> CH3COOC2H5(aq) + H2O(l)}
\item \ce{Al(OH)3(s) + OH^-(aq) <=> Al(OH)4^-(aq)}
\end{itemize}
\\\hline
導電度 & 液相者利用導電度計以電極法測量,固相者利用檢流計與電阻測量 & 只適用於反應前後有電荷載子(如離子、自由電子、電洞)濃度變化者 & 
\begin{itemize}
\item \ce{H2O2(aq) + 2HI(aq)\text{(解離)} -> 2H2O(l) + I2(s)}
\item \ce{H2SO4(aq)\text{(解離)} + Ba(OH)2(aq)\text{(解離)} -> BaSO4(s) + 2H2O(l)}
\end{itemize}
\\\hline
還原電位 & 與參考電極連接以電極法測量 & 只適用於反應物與生成物還原電位不同者 & 
\begin{itemize}
\item \ce{H2O2(aq) + 2HI(aq) <=> 2H2O(l) + I2(s)}
\item 
\ce{2[Fe(CN)6]^{3-}(aq) + 2S2O3^{2-}(aq) <=> 2[Fe(CN)6]^{4-}(aq) + S4O6^{2-}(aq)}
\end{itemize}
\\\hline
旋光性 & 使偏振光通過之並測量通過後光的偏振性 & 只適用於一當量反應物與一當量生成物旋光性不同者 & 蔗糖(aq)(左旋光) + \ce{H2O(l) ->} 葡萄糖(aq)(右旋光) + 果糖(aq)(左旋光)
\\\hline
光譜 & 利用光譜儀測量光譜 & 基於分子帶狀特徵光譜 & 幾乎全部反應
\\\hline
黏滯性 & 利用黏度計測量黏度 & 只適用於反應物與生成物黏度不同者 & 許多聚合反應
\\\hline
\end{longtable}\FloatBarrier
\ssc{實驗舉隅}
\subsubsection{二氧化氮雙聚反應實驗}
可逆放熱反應:
\[\ce{2NO2(g)\tx{(紅棕)} <=> N2O4(g)\tx{(透明)}}\]
\ben
\item 玻璃大試管中置濃硝酸並加入銅片或銅線,上雙孔橡皮塞兩孔各接一玻璃彎管,兩玻璃彎管上各接橡皮管,一橡皮管接至一漏斗之窄端(或接一雙孔橡皮塞試管再以另一孔之橡皮管接至一漏斗之窄端),並將該漏斗之寬端倒置於裝有 1M NaOH 的燒杯中(以增加氣體與 NaOH 的接觸面積),另一橡皮管接至一注射筒之細端。各接口均應密閉。
\item 銅加濃硝酸發生反應:
\[\ce{Cu(s) + 4HNO3(aq) -> Cu(NO3)2(aq) + 2NO2(g) + 2H2O(l)}\]
\item NaOH 以酸鹼中和吸收溢出的 \ce{NO2}:
\[\ce{2NO2(g) + 2NaOH(aq) -> NaNO2(aq) + NaNO3(aq) + H2O(l)}\]
\item 靜置一陣子待大試管中空氣均排除而充滿紅棕色二氧化氮氣體後,以三相同有活塞之注射筒分別吸取大試管中的二氧化氮氣體,其一為壓力實驗實驗組,另一為壓力實驗對照組,再一為溫度實驗實驗組,立即用有凹槽之橡皮塞塞住管口。另取等量空氣於另一相同注射筒為溫度實驗對照組,立即用有凹槽之橡皮塞塞住管口。
\item 取一二氧化氮注射筒與一空氣注射筒施以壓力或拉力使至相同高度,壓力加大,反應向右,分壓均增加,顏色變深;壓力減小,反應向左,分壓均減少,顏色變淺。取兩二氧化氮注射筒,其一升溫或降溫,溫度降低,反應向右,顏色變淺;溫度升高,反應向左,顏色變深。
\een
\subsubsection{氯化亞鈷實驗}
可逆吸熱反應:
\[\ce{[Co(H2O)6]^{2+}(aq)\text{(粉紅)} + 4Cl^-(aq) <=> [CoCl4]^{2-}(aq)\text{(藍)} + 6H2O(l)}\]
\ben
\item 取 0.25M 氯化亞鈷溶液 25 mL,分別加 2 mL 於三試管為對照組;第一試管粉紅;第二試管加入 10 克無水氯化鈣,變藍色;第三試管,加入 2 克無水氯化鈣,變紫色。
\item 配置同前述第三試管之試管五支為實驗組,第一試管加入更多無水氯化鈣,變藍;第二試管加熱,變藍;第三試管泡冰水浴,變粉紅;第四試管加水,變粉紅;第五試管加硝酸銀溶液,變粉紅並產生白色\ce{AgCl}沉澱。
\een
\subsubsection{硫氰化鐵比色法測定平衡常數實驗}
可逆反應:
\[\ce{Fe^{3+}\tx{(淡黃)} + SCN^-\tx{(透明)} <=> FeSCN^{2+}\tx{(血紅)}}\]
兩相同口徑比色管所盛溶液之透光度相同時,溶液高度反比於硫氰化鐵的濃度。
\ben
\item 取 0.002 M \ce{KSCN} 5 mL + 0.2 M \ce{Fe(NO3)3} 5 mL 於一號燒杯,以安全吸球與分度吸量管取 5 mL 置於一號比色管,作為 0.001 M \ce{FeSCN^{2+}} 標準溶液。(可假設全數反應乃因$K_c$夠大;用硝酸鐵溶液乃為避免\ce{Fe(OH)3}沉澱。)
\item 取一號燒杯溶液 5 mL,於二號燒杯中加蒸餾水稀釋至 25 mL,以安全吸球與分度吸量管取 5 mL 置於二號比色管。
\item 重複上一步驟方法,取前一號燒杯溶液稀釋後取 5 mL 置於三至五號比色管。
\item 可在各比色管滴加等量約一滴的硝酸以避免\ce{Fe(OH)3}沉澱。
\item 取一號比色管與二號比色管均包黑紙置於比色法燈具光源上方,光自比色管底部向上照射,調整一號比色管之溶液高度,直到兩管透光度相同,記錄之。
\item 重複上一步驟方法,依序將二號比色管替換為三至五號比色管。
\item 由溶液高度計算濃度,進而求出平衡常數,通常約在50至200。
\een
\subsubsection{氫氧化鈉滴定酸實驗}
標定:
\begin{enumerate}
\item 精稱鄰苯二甲酸氫鉀 0.2 g,倒入錐形瓶中,加入 50 mL 水,使完全溶解,再加入 2-3 滴酚酞,搖盪使充分混勻。
\item 以分度吸量管取 100 mL 0.1 M 氫氧化鈉水溶液於燒杯中。
\item 以氫氧化鈉水溶液潤洗滴定管。
\item 以漏斗將氫氧化鈉水溶液加入滴定管中,記錄讀數。
\item 錐形瓶置於滴定管尖口下方,以氫氧化鈉水溶液滴定,反應:
\[\tx{鄰苯二甲酸氫鉀(aq)} + \ce{NaOH(aq) ->}\tx{鄰苯二甲酸鉀鈉(aq)} + \ce{H2O(l)}\]
\item 直到溶液變為粉紅並 30 秒不褪色,記錄讀數。
\item 重複以上步驟,得氫氧化鈉水溶液當量濃度,取平均。
\end{enumerate}
滴定:
\begin{enumerate}
\item 以分度吸量管取未知酸 10 mL 加入錐形瓶,再加入 2-3 滴酚酞,搖盪使充分混勻。
\item 錐形瓶置於滴定管尖口下方,以氫氧化鈉水溶液滴定,直到溶液變為粉紅並 30 秒不褪色,記錄讀數。
\item 重複以上步驟,得未知酸當量濃度,取平均。
\end{enumerate}
\sssc{酸性過錳酸鉀滴定測定草酸鎂溶解度積常數實驗}
標定:
\begin{enumerate}
\item 精稱草酸鈉 0.2 g,倒入錐形瓶中,加入 50 mL 60-75°C 熱蒸餾水,使完全溶解。因草酸鈉與過錳酸鉀反應在室溫下較慢,故用熱水以提高反應速率,但溫度不可高於 90°C 以免草酸根或過錳酸根受熱分解,若時間較多亦可用室溫水。
\item 以兩分度吸量管分別取 1 mL 3.0 M 硫酸水溶液與 100 mL 0.01 M 過錳酸鉀水溶液於一燒杯中,搖盪使充分混勻。
\item 以酸性過錳酸鉀水溶液潤洗滴定管。
\item 以漏斗將酸性過錳酸鉀水溶液加入滴定管中,記錄讀數。
\item 錐形瓶置於滴定管尖口下方,以酸性過錳酸鉀水溶液滴定,反應:
\[\ce{2MnO4^-(aq) + 5C2O4^{2-}(aq) + 16H+(aq) -> 2Mn^{2+} + 10CO2(g) + 8H2O(l)}\]
\item 直到溶液變為紫紅並維持 30 秒,記錄讀數。
\item 重複以上步驟,得酸性過錳酸鉀水溶液當量濃度,取平均。
\end{enumerate}
滴定
\ben
\item 將過量草酸鎂固體粉末(難溶)加入水中並攪拌。
\item 以離心機分離固體與液體,小心取出上層澄清液,即飽和草酸鎂水溶液。
\item 以兩分度吸量管分別取 10 mL 飽和草酸鎂水溶液與 20 mL 熱蒸餾水於一錐形瓶中。
\item 錐形瓶置於滴定管尖口下方,以酸性過錳酸鉀水溶液滴定。
\item 直到溶液變為紫紅並維持 30 秒,記錄讀數。
\item 重複以上步驟,得飽和草酸鎂水溶液當量濃度,取平均,計算草酸鎂溶解度積常數。
\een
\subsubsection{2-羥基苯甲酸(柳酸)與乙酐製備 2-乙醯氧基苯甲酸(乙醯柳酸/阿斯匹靈)實驗}
\tb{製備:}
\begin{enumerate}
\item 取乾燥試管一支,取 1 g 柳酸(限量試劑)與 2 mL 乙酐(既是反應物也是溶劑)加入其中。乙酐觸碰到人體應視同強酸處理,立刻以大量水沖洗。
\item 在試管中滴入 3 滴濃硫酸(催化劑),搖盪混合均勻。
\item 取一燒杯,裝入適量熱水,將試管直立置於其中,試管內的液體須完全沒入燒杯液面下,但不可使燒杯的水流入試管,燒杯水量使試管不會漂浮為宜。
\item 取一錐形瓶,加入 15 mL 蒸餾水。
\item 取一燒杯,裝入適量水與冰塊作為冰水浴,將錐形瓶直立置於其中,錐形瓶內的液體須完全沒入燒杯液面下,但不可使燒杯的水流入錐形瓶,燒杯水量使錐形瓶不會漂浮為宜。
\item 不斷以玻棒攪拌並搖盪前述試管約 10 分鐘,若燒杯水溫降低須更換熱水,直到固體完全溶解:
\bma
& \tx{\ce{C6H4(OH)(COOH)(s/aq) + (CH3CO)2O(l)}}\\
\tx{\ce{->[\ce{H2SO4}]}} & \tx{\ce{C6H4(OCOCH3)(COOH)(aq) + CH3COOH(aq)}}
\eam
\item 趁熱於試管中加入 2 mL 蒸餾水,分解剩餘的乙酐。
\item 將試管靜置於試管架上冷卻至室溫。
\item 將試管內物質倒入前述錐形瓶中,另取蒸餾水 1 mL 沖洗試管,洗液亦倒入前述錐形瓶中,錐形瓶仍置於冰水浴靜置,若燒杯水溫升高須添加冰塊。
\item 不斷以玻棒攪拌並搖盪錐形瓶,使白色沉澱充分析出(因阿斯匹靈難溶於水、乙酸易溶於水)。
\item 減壓抽濾錐形瓶中物質至抽濾瓶中,附著於瓶壁的固體可用濾液沖洗至抽濾漏斗內,盡量將水分抽乾,收集濾紙上的白色固體粉末,此即阿斯匹靈粗產物,抽濾瓶中液體倒入有機廢液筒中。
\end{enumerate}
\tb{純化:}
\begin{enumerate}
\item 以刮勺將濾紙上的固體移入一 100 mL 燒杯,慢慢加入至多 10 mL 飽和碳酸氫鈉水溶液,加入過程中不斷以玻棒攪拌,直到不再有氣泡產生,表示阿斯匹靈完全溶解:
\bma
& \tx{\ce{C6H4(OCOCH3)(COOH)(s/aq) + NaHCO3(aq)}}\\
\tx{\ce{->}} & \tx{\ce{C6H4(OCOCH3)(COO$^-$)Na$^+$(aq) + H2O(l) + CO2(g)}}
\eam
\item 簡單過濾燒杯中物質至一 50 mL 錐形瓶中,附著於杯壁的固體用少許冰水沖洗至漏斗內,濾紙上的固體丟棄。
\item 將 10 mL 6 M 鹽酸以玻棒倒入裝有濾液的錐形瓶中(阿斯匹靈在酸中溶解度降低),加入過程中不斷以玻棒攪拌,使白色沉澱析出:
\[\ce{C6H4(OCOCH3)(COO$^-$)Na$^+$(aq) + HCl(aq) -> C6H4(OCOCH3)(COOH)(s) + NaCl(aq)}\]
\item 將錐形瓶放入前述冰水浴中,若燒杯水溫升高須添加冰塊,靜置數分鐘待沉澱完全。
\item 減壓抽濾錐形瓶中物質至抽濾瓶中,附著於瓶壁的固體可用濾液沖洗至抽濾漏斗內,盡量將水分抽乾,收集濾紙上的白色固體粉末,此即阿斯匹靈,抽濾瓶中液體倒入有機廢液筒中。
\item 取一乾燥錶玻璃,測量其質量。
\item 以刮勺將濾紙上的固體移至錶玻璃上攤開,放入烘箱 100°C 烘乾,每次烘乾完畢連同錶玻璃一同測量質量,烘乾後較烘乾前質量變化小於 0.01 g 視為完全乾燥,所得白色固體粉末為阿斯匹靈最終產物。可由冷卻至室溫後其連同錶玻璃質量減去錶玻璃質量得產物質量並計算產率。不可食用。
\end{enumerate}
\tb{檢驗:}
\ben
\item 取二試管,分別加入 0.05 g 最終產物與 0.05 g 柳酸。
\item 各加入 1 mL 酒精(作為溶劑,使為均勻相)與 1 滴 黃色\ce{FeCl3(aq)},搖晃兩試管使混合均勻。
\item 柳酸與鐵(III)離子形成紫色錯合物,阿斯匹靈則否。
\een
\sssc{氧化還原電池實驗}
\ben
\item 下使用銅棒電極者,實驗前先打磨銅棒刮除銅鏽(如有)。
\item 下使用石墨棒電極者,實驗後須打磨石墨棒刮除金屬(如有)。
\item 取 U 型管,其中加入 0.1 M \ce{KNO3(aq)} ,並將兩開口以溼棉花塞住,作為鹽橋。每組電池實驗過後,若實驗足夠快燒杯液體未滲入 U 型管內則僅須更換棉花,否則須更換其中溶液。
\item \tb{鋅銅電池銅電極}:取二 100 mL 乾淨燒杯 A、B,各加入 0.1 M \ce{ZnSO4(aq)}、\ce{CuSO4(aq)},分別插入鋅棒與銅棒,兩棒以帶鱷魚夾導線分別與伏特計負、正極連接,鹽橋倒置導通兩燒杯,測量電壓,應偏向 B。
\item \tb{鋅銅電池石墨電極}:同上但將銅棒換為石墨棒。
\item \tb{鋅銀電池銀電極}:同上但將\ce{CuSO4(aq)}換為\ce{AgNO3(aq)}、將石墨棒換為銀棒。
\item \tb{鋅銀電池石墨電極}:同上但將銀棒換為石墨棒。
\item \tb{銅銀電池銀電極}:同上但將\ce{ZnSO4(aq)}換為\ce{CuSO4(aq)}、將鋅棒換為銅棒、將石墨棒換為銀棒。
\item \tb{銅銀電池石墨電極}:同上但將銀棒換為石墨棒。
\item \tb{鋅濃度電池}:同上但將 0.1 M \ce{CuSO4(aq)} 換為 0.01 M \ce{ZnSO4(aq)}、將\ce{AgNO3(aq)}換為\ce{ZnSO4(aq)}、將銅棒換為鋅棒、將銀棒換為鋅棒。
\een
\sssc{碘化鉀電解實驗}
\tb{電解}:U 型管中裝入碘化鉀水溶液,兩管口各插入石墨棒作為電極,施加 12 伏特直流電:
\bit
\item 正/陽極半反應:\ce{3I^-(aq) -> I3^-(aq) + 2e-}或\ce{2I^-(aq) -> I2(aq) + 2e-},又\ce{I^-(aq) + I2(aq) <=> I3^-(aq)},\ce{I3-}呈棕色。
\item 負/陰極半反應:\ce{H2O(l) + 2e- -> H2(g) + 2OH^-(aq)},釋出氫氣,並增加 pH 值。
\item U 型管中央兩極界面附近反應:\ce{3I3^-(aq) + 6OH^-(aq) -> 8I^-(aq) + IO3^-(aq) + 3H2O(l)}或\ce{3I2(aq) + 6OH^-(aq) -> 5I^-(aq) + IO3^-(aq) + 3H2O(l)},產物均無色,並減少 pH 值,可見清楚棕色與無色交介面。
\eit
\tb{檢驗}:
\bit
\item 取陽極附近溶液加入環己烷,溶液分為兩層,下層為無色或淡紫色的碘化鉀水溶液,上層為紫色的碘環己烷溶液。
\item 取陽極附近溶液加入澱粉,呈深藍色。
\item 取陰極附近溶液加入酚酞,變為粉紅色。
\item 取陰極附近溶液加入氯化鐵(III),產生紅棕色\ce{Fe(OH)3}沉澱。
\item 在陰極管口附近點火,氫氣燃燒產生淡藍色火焰並有爆鳴聲。
\eit
\sssc{鍍鋅實驗}
\ben
\item 被鍍物氧化電位不可高於鋅。將被鍍物以砂紙打磨,再以蒸餾水沖洗,再以丙酮沖洗。若上汙垢、鏽蝕、氧化物膜等過多,且被鍍物耐酸或鹼,可酸浸或鹼浸。
\item 取一燒杯配置 0.5 M 碳酸鋅、0.1 M 氫氧化鈉水溶液。鹼性條件係為避免陰極產生氫氣。
\item 燒杯中插入鋅片與銅片作為陽極與陰極,兩者不可接觸,分別連接直流電源正極與負極,施加 12 伏特直流電:
\bit
\item 正/陽極半反應:\ce{Zn(s) -> Zn^{2+}(aq) + 2e-}
\item 負/陰極半反應:\ce{Zn^{2+}(aq) + 2e- -> Zn(s)}
\eit
\een
\sssc{銀鏡反應(Silver mirror reaction)實驗}
\tb{多侖試劑(Tollens' reagent)}\ce{[Ag(NH3)2]NO3(aq)}之配置:
\begin{enumerate}
\item 以試管刷取出試管內壁汙垢。
\item 滴入 0.6 M \ce{AgNO3(aq)} 3.0 mL。
\item 滴入 2.5 M \ce{NaOH(aq)} 2至3滴,形成 \ce{Ag2O(s)}沉澱。
\item 逐滴滴入 2.0 M \ce{NH3(aq)} 搖晃至沉澱完全溶解為二氨銀錯離子 \ce{[Ag(NH3)2]+}。
\item 多侖試劑必須隨配隨用而不可久置,且含有多侖試劑的廢液須加水稀釋方可倒棄,均為防止生成氮化銀 \ce{Ag3N} 、疊氮化銀 \ce{AgN3} 等不穩定而易爆炸之物質。
\end{enumerate}

\tb{銀鏡反應}:在多侖試劑中滴入 10\% 葡萄糖水溶液 5-7 滴,隔水加熱約一分鐘且溫度控制在低於 70°C,發生銀鏡反應,多侖試劑的銀離子還原析出銀於試管壁上,屬無電電鍍:
\bma
& \tx{\ce{C5H11O5CHO(aq) + 2[Ag(NH3)2]+(aq) + 3OH^-(aq)}}\\
\tx{\ce{->}} & \tx{\ce{C5H11O5COO^-(aq) + 2Ag(s) + 4NH3(aq) + 2H2O(l)}}
\eam
\sssc{奈米硫粒子製備實驗}
\ben
\item A 溶液:1.0 M 硫代硫酸鈉水溶液 2 毫升 + 稀釋清潔劑 5 滴,再加水至 20 mL。清潔劑提供界面活性劑幫助奈米硫粒子分散,以延長奈米硫粒子膠體溶液存在時間。
\item B 溶液:2.0 M 鹽酸 2 mL,再加水至 25 mL。
\item 於燒杯中混合兩溶液:
\[\ce{S2O3^{2-}(aq) + 2H+(aq) -> SO2(g) + H2O(l) + S(s)}\]
\[\ce{S2O3^{2-}(aq) + 2H+(aq) -> H2SO3(aq) + S(s)}\]
\item 以雷射筆照射,初始無光徑,後奈米硫粒子形成膠體溶液,有廷得耳效應,出現光徑,最後硫粒子形成大顆粒沉澱,光徑消失。
\item 硫固體應過濾回收,液體可稀釋後排放於水槽。
\een
\sssc{界面活性劑效應實驗}
\ben
\item 取二 100 mL 燒杯 A、B,各加入 50°C、30 mL 熱水。
\item 在 A、B 燒杯中分別加入0.5 g(約一小刮勺平匙)十二烷基硫酸鈉與肥皂絲(硬脂肪酸鈉) ,緩緩攪拌至燒杯內物質與水完全互溶,配置出 A 溶液與 B 溶液。
\item 取 7 支乾淨試管 1 至 7,1 試管中加入 3 mL 蒸餾水,其餘試管中偶數編號者中加入 3 mL A 溶液、奇數編號者中加入 3 mL B 溶液,靜置兩分鐘。
\item 取乾淨玻棒沾取 1、2、3 溶液,以廣用試紙檢測其酸鹼性。
\item 分別搖晃 1、2、3 試管,觀察其起泡情況。
\item 在 4、5 試管中各加入 0.1 M \ce{CaCl2(aq)} 10 滴,蓋上橡皮塞充分搖晃混合,觀察混合初與靜置三分鐘後的情形。
\item 在 6、7 試管中各加入 1 M 鹽酸水溶液 10 滴,蓋上橡皮塞充分搖晃混合,觀察混合初與靜置三分鐘後的情形。
\item 在 1 至 7 試管中各加入 1 滴油性墨水,蓋上橡皮塞充分搖晃混合,觀察混合初與靜置三分鐘後的情形。
\item 以清潔劑將試管內油墨與固體殘留等刷洗乾淨後,以大量清水沖洗排放於水槽。
\een
\sssc{酸鹼中和反應熱測定實驗}
\ben
\item 洗淨、擦乾卡計,測量卡計鋼杯質量。
\item 取二量筒分別量取 40 mL 2.0 M 鹽酸與 40 mL 2.0 M 氫氧化鈉水溶液,測量溫度。
\item 將氫氧化鈉水溶液倒入杯中。
\item 備妥插有溫度計與攪拌棒的卡計杯蓋。
\item 將鹽酸迅速倒入杯中,立即蓋上杯蓋並插入溫度計與攪拌棒。
\item 每隔數秒記錄一次溫度,直到讀數連續 30 秒內溫度變化小於 0.1 K。以溫度最大值為終止溫度。
\item 靜置待恢復室溫。
\item 取出杯,測量杯連同其中溶液之質量。
\item 將杯中溶液以漏斗與玻棒倒入量筒中測量其體積。
\een
\sssc{硝酸鉀固體溶解熱測定實驗}
\ben
\item 洗淨、擦乾卡計,測量卡計鋼杯質量。
\item 取一量筒量取 50 mL 逆滲透水,測量溫度。
\item 以鋼杯為容器用天平稱取 5 公克硝酸鉀固體,並測量硝酸鉀固體與鋼杯之總重。
\item 備妥插有溫度計與攪拌棒的卡計杯蓋。
\item 將逆滲透水迅速倒入杯中,立即蓋上杯蓋並插入溫度計與攪拌棒。
\item 每隔數秒記錄一次溫度,直到讀數連續 30 秒內溫度變化小於 0.1 K。以溫度最小值為終止溫度。
\item 靜置待恢復室溫。
\item 取出杯,測量杯連同其中溶液之質量。
\item 將杯中溶液以漏斗與玻棒倒入量筒中測量其體積。
\een
\sssc{乙醇與水非理想溶液與乙酸乙酯與乙酸丙酯理想溶液實驗}
\ben
\item 取六個量筒,以一分度吸量管依序分別加入 1 mL、2 mL、4 mL、6 mL、8 mL 與 9 mL 乙醇。
\item 以另一分度吸量管依序分別加入 9 mL、8 mL、6 mL、4 mL、2 mL 與 1 mL 蒸餾水。
\item 搖晃各量筒使溶液混合均勻。
\item 測量各量筒內溶液體積與溫度。
\item 重複以上步驟,但將乙醇與蒸餾水換成乙酸乙酯與乙酸丙酯。因苯與甲苯劇毒,故用乙酸乙酯與乙酸丙酯。
\een
\sssc{凝固點下降實驗}
\ben
\item 取一小試管,以分度吸量管加入 5 mL 純水。
\item 用雙孔分別插入溫度計與銅絲攪拌器的雙孔橡皮塞塞住小試管口。
\item 將小試管套入一大試管中。
\item 取一 500 mL 燒杯,置入 3:1 之冰與食鹽混合物作為冷劑。
\item 將雙層試管插入冰鹽冷劑中,使整套裝置盡量被冷劑包覆。
\item 以攪拌器輕輕上下攪拌小試管內物質,每隔 10 秒記錄一次溫度與凝固情形,直到完全凝固。若看不清可於大試管旁置一空試管,透過之觀察。常有過冷現象,先下降至凝固點以下且不凝固,而後開始凝固並溫度上升,達到凝固點後繼續凝固,直到完全凝固開始繼續降溫,應以凝固過程溫度線的延長線與凝固前溫度線的交點溫度為凝固點。
\item 將冰鹽冷劑換成 1:4 之冰與水,雙層試管不變。
\item 以攪拌器輕輕上下攪拌小試管內物質,每隔 10 秒記錄一次溫度與熔化情形,直到完全熔化。若看不清可於大試管旁置一空試管,透過之觀察。
\item 用天平精稱 600 mg 尿素,加入小試管純水中,以攪拌器攪拌直到完全溶解。
\item 重複第二至第八步驟。凝固過程中,因濃度愈高凝固點愈低,凝固者濃度低於未凝固者,使未凝固者濃度不斷提升,溫度(凝固點)不斷下降。
\een
\sssc{碘鐘反應(Iodine clock reaction)}
\tb{儀器}:
\bit
\item 天平 1 個。
\item 試管 10 支。
\item 50 mL 錐形瓶 1 個。
\item 100 mL 小燒杯 1 個。
\item 1 L 大燒杯 3 個。
\item 0°C-100°C 溫度計 1 支。
\item 安全吸球與分度吸量管 2 組,一裝試劑 A、一裝試劑 B,試劑配置時取其一裝硫酸,配置完成後洗淨並以試劑 A 或 B 潤洗以用於之。
\item 碼錶 1 個,測量時以試管之溶液加入錐形瓶為開始,溶液顯現出藍色為結束。
\eit

\tb{試劑配置}:
\bit
\item 試劑 A — 過量\ce{KIO3(aq)}:以天平精稱 4.28 g 碘酸鉀白色固體,加入大燒杯中,加水配成 1 L。
\item 試劑 B — 限量\ce{NaHSO3(aq) + H2SO4(aq) + \text{直鏈澱粉}}:
\ben
\item 以天平稱取 4 g 直鏈澱粉,加入一小燒杯中,加水配成約 20 mL,攪拌均勻。
\item 取一大燒杯,裝約 900 mL 熱水,將澱粉水溶液倒入其中。
\item 以酒精燈間接加熱大燒杯,煮沸 3 至 5 分鐘後,關火,靜置冷卻至室溫。
\item 以天平精稱 0.19 g 焦亞硫酸鈉/偏二亞硫酸鈉白色固體,加入大燒杯中:
\[\ce{Na2S2O5(s) + H2O(l) -> 2Na+(aq) + 2HSO3^-(aq)}\]
\item 以分度吸量管取 5 mL 1 M 硫酸水溶液,加入大燒杯中。
\item 加水於大燒杯配成 1 L。
\een
\eit

\tb{注意事項}:
\bit
\item 試劑 B 提供之亞硫酸氫根須為限量試劑,即初始時$\frac{[\ce{IO3^-}]}{[\ce{HSO3^-}]}>\frac{1}{3}$,因為仍有亞硫酸氫根存在時會還原碘:
\[\ce{I2(aq) + HSO3^-(aq) + H2O(l) -> 2I^-(aq) + HSO4^-(aq) + 2H+(aq)}\]
使碘無法累積,藍色錯合物無法產生。
\item 試劑 B 不可久置,因\ce{NaHSO3(aq)}在空氣中不穩定而易被氧化:
\[\ce{2HSO3^-(aq) + O2(g) -> HSO4^-(aq)}\]
\item 反應溫度須在 45°C 以下,否則藍色錯合物難以生成。
\item 實驗主反應:
\[\ce{IO3^-(aq) + 3HSO3^-(aq) -> I^-(aq) + 3SO4^{2-}(aq) + 3H+ (aq)}\]
的反應速率定律式為$-\dv{[\ce{HSO3-}]}{t}=k[\ce{HSO3-}][\ce{IO3-}]$,速率適中,故碘鐘反應為實驗室常用秒錶反應(clock reaction)/化學鐘(chemical clock)實驗,以$[\ce{HSO3-}]$為操縱變因,以溶液變為藍色的時間(即消耗完$[\ce{HSO3}]$的時間)為應變變因。
\item 試劑 B 之硫酸當量影響溶液酸鹼,從而影響$[\ce{HSO3-}]$並影響反應速率,可:
\ben
\item 取二試管,分別以分度吸量管加入 10 mL 試劑 A 與 B。
\item 將兩試管之溶液同時倒入錐形瓶,儘速搖盪錐形瓶使混合均勻,測量其變為藍色的時間,若小於 5 秒代表硫酸當量過多、大於 5 分鐘代表硫酸當量過少。
\een
\eit

\tb{反應:}
\ben
\item 混合初期亞硫酸氫根較慢地還原碘酸根:
\[\ce{IO3^-(aq) + 3HSO3^-(aq) -> I^-(aq) + 3SO4^{2-}(aq) + 3H+ (aq)}\]
\item 接著快速發生反自身氧化還原反應:
\[\ce{5I^-(aq) + IO3^-(aq) + 6H+ (aq) -> 3I2(s) + 3H2O(l)}\]
\item 亞硫酸氫根耗盡後極快速發生反應:
\[\ce{I2 + \text{直鏈澱粉} -> \text{藍色錯合物}}\]
\een

\tb{實驗步驟}:
\ben
\item 取五試管,各以分度吸量管加入 10 mL 試劑 B。
\item 取五試管 A$_1$ 至 A$_5$,分別以分度吸量管加入 10 mL、8 mL、6 mL、4 mL 與 2 mL 試劑 A。
\item 在試管 A$_2$ 至 A$_5$,分別以分度吸量管加入 2 mL、4 mL、6 mL 與 8 mL 蒸餾水。
\item 將試管 A$_1$ 與一支試劑 B 試管之溶液同時倒入錐形瓶,儘速搖盪錐形瓶使混合均勻,測量其變為藍色的時間。完成後洗淨錐形瓶。
\item 重複上一步驟依序測試試管 A$_2$ 至 A$_5$。
\item 將該五原先裝試劑 B 之試管,各以分度吸量管加入 10 mL 試劑 B。
\item 洗淨試管 A$_1$ 至 A$_5$,各以分度吸量管加入 10 mL 試劑 A,令為 A$_6$ 至 A$_{10}$。
\item 在一大燒杯中裝入約半滿的自來水,並加入適量冰塊,調控水溫約 5°C。
\item 將試管 A$_6$、一支試劑 B 試管與錐形瓶浸入其中,試管內的液體須完全沒入燒杯液面下,但不可使燒杯的水流入試管,燒杯水量使試管與錐形瓶不會漂浮為宜,靜置約 5 分鐘使試管與錐形瓶與燒杯液體同溫,測量其溫度。
\item 將兩試管之溶液同時倒入錐形瓶,錐形瓶仍在大燒杯中,儘速搖盪錐形瓶使混合均勻,測量其變為藍色的時間。完成後洗淨錐形瓶。
\item 重複前三步驟,調控水溫約 10°C、15°C、20°C、25°C,依序測試試管 A$_7$ 至 A$_{10}$。
\een
\sssc{醇鈉實驗}
\ben
\item 取乾淨試管 5 支,分別裝入約 1 mL 甲醇、乙醇、1-丙醇、2-丙醇、2-甲基-2-丙醇。
\item 以鑷子夾取金屬鈉條置於玻璃板上,以小刀切下 5 塊約 5 mm $\times$ 5 mm $\times$ 5 mm 的鈉塊,以鑷子夾取分別加入 5 支試管。此步驟應儘快,否則鈉將被空氣氧化為氧化鈉。
\item 反應速度:甲醇>乙醇>1-丙醇>2-丙醇>2-甲基-2-丙醇。
\een
\sssc{醇、醛、酮與酸性過錳酸鉀反應實驗}
\ben
\item 取乾淨試管,裝入 10 滴甲醇、2 滴硫酸水溶液與 1 滴過錳酸鉀水溶液,充分搖盪 30 秒,觀察是否褪色。
\item 將甲醇分別換為乙醇、1-丙醇、2-丙醇、2-甲基-2-丙醇、甲醛水溶液、乙醛水溶液、丙醛、丁醛、丙酮、丁酮重複以上步驟。
\item 甲醇、乙醇、1-丙醇、2-丙醇、甲醛水溶液、乙醛水溶液、丙醛、丁醛會褪色,2-甲基-2-丙醇、丙酮、丁酮則否。
\een
\sssc{醇、醛、酮與裴琳試液反應實驗}
\ben
\item 取乾淨試管,裝入 1 mL 甲醇與 1 mL 裴琳試液。
\item 將甲醇分別換為乙醇、1-丙醇、2-丙醇、2-甲基-2-丙醇、甲醛水溶液、乙醛水溶液、丙醛、丁醛、丙酮、丁酮重複以上步驟。
\item 取一 250 mL 大燒杯,裝入適量熱水,將所有試管一起直立置於其中,試管內的液體須完全沒入燒杯液面下,但不可使燒杯的水流入試管,燒杯水量使試管不會漂浮為宜。
\item 將燒杯置於抽氣罩下抽氣,搖動各試管,小心避免燙傷,觀察各試管溶液的變化。
\item 甲醛水溶液、乙醛水溶液、丙醛、丁醛會產生磚紅色氧化亞銅沉澱與藍色褪色,甲醇、乙醇、1-丙醇、2-丙醇、2-甲基-2-丙醇、丙酮、丁酮則否。
\een
\end{document}