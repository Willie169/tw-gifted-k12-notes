\documentclass[a4paper,12pt]{report}
\setcounter{secnumdepth}{5}
\setcounter{tocdepth}{3}
\newcounter{ZhRenew}
\setcounter{ZhRenew}{1}
\newcounter{SectionLanguage}
\setcounter{SectionLanguage}{1}
\input{/usr/share/latex-toolkit/template.tex}
\begin{document}
\title{單變數實多項式函數}
\author{沈威宇}
\date{\temtoday}
\titletocdoc
\section{單變數實多項式函數}
\subsection{多項式(Polynomial)}
指由多個項(term)組成的代數表達式,每個項是常數(稱該項係數(coefficient))與零或正整數(為零則該項稱常數項(constant term))個變數的非負整數冪次的乘積。 
\ssc{單變數實多項式函數定理}
下$f(x)$、$g(x)$、$q(x)$、$r(x)$均為$x$的實多項式,$a$、$b$為非零實數。
\sssc{領導係數(leading coefficient)}
指單變數多項式的最高次項係數。
\sssc{最高次項次數}
\[\deg(af(x)\cdot g(x))=\deg(f(x))+\deg(g(x))\]
\[\deg(af(x)+bg(x))\leq\max(\deg(f(x)),\deg(g(x)))\]
\sssc{除法原理/商定則(Quotient rule)}
恰有一組$q(x)$、$r(x)$滿足:
\[f(x)=g(x)\cdot q(x)+r(x),\quad\deg(r(x))<\deg(g(x))\lor r(x)=0\]
稱被除式$f(x)$除以除式$g(x)$的商式為$q(x)$、餘式為$r(x)$。若$r(x)=0$則$q(x)$為$f(x)$的因式,$f(x)$為$q(x)$的倍式。
\sssc{餘式定理}
$f(x)$除以$(ax+b)$的餘式為$f\qty(\frac{b}{a})$。
\sssc{因式定理}
$f\qty(\frac{b}{a})=0$若且惟若$ax+b$為$f(x)$的因式。
\sssc{恆等定理}
兩$n$次多項式$f(x)$與$g(x)$,若存在$n+1$個相異數$a$使得$f(a)=g(a)$,則$f(x)=g(x)$。
\subsection{零函數}
\[f(x)=0\]
\subsection{一元一次函數}
$a\neq 0$
\subsubsection{一般式、標準式}
\[f(x)=ax+b\]
其中$a$稱斜率,$b$稱$y$截距。
\subsubsection{線性函數(Linear function)}
零函數與一元一次函數合稱線性函數。
\subsubsection{圖形特徵}
\begin{itemize}
\item 直線。
\item 二階旋轉對稱點:圖形上任一點。
\item 沒有反曲點。
\end{itemize}
\sssc{根數}
一實根。
\subsection{一元二次函數}
$a\neq 0$
\subsubsection{一般式}
\[f(x)=ax^2+bx+c\]
\subsubsection{標準式}
\[f(x)=a(x-h)^2+k\]
其中$ (h,k)=\qty(-\frac{b}{2a},-\frac{b^2-4ac}{4a})$
\subsubsection{判別式}
\[\Delta=b^2-4ac\]
\subsubsection{根}
\[x =\frac{-b\pm\sqrt{b^2-4ac}}{2a}=\frac{-b\pm\sqrt{\Delta}}{2a}\]
\subsubsection{圖形特徵}
\begin{itemize}
\item 拋物線。
\item 頂點、極值點、最值點:$(h,k)$。
\item 對稱軸:$x=h$。
\item 開口:$a$為正,開口向上;$a$為負,向下。$|a|$愈大,開口愈小。
\item 沒有反曲點。
\end{itemize}
\sssc{根數}
\begin{itemize}
\item $\Delta>0$:二相異實根。
\item $\Delta=0$:一實根且為二重根。
\item $\Delta<0$:無實根,二相異共軛虛根,$a>0$則$f(x)>0$,$a<0$則$f(x)<0$。
\end{itemize}
\sssc{根與係數關係}
令根$\alpha$、$\beta$
\[\alpha+\beta=-\frac{b}{a}\]
\[\alpha\beta=\frac{c}{a}\]
\subsection{一元三次函數}
$a\neq 0$
\subsubsection{一般式}
\[f(x)=ax^3+bx^x+cx+d\]
\subsubsection{標準式}
\[f(x)=a(x-h)^3+p(x-h)+k\]
其中$(h,p,k)=\qty(-\frac{b}{3a},c-\frac{b^2}{3a},\frac{2b^3-9abc}{27a^2}+d)$
\subsubsection{判別式}
\[\Delta = 18abc - 4a^3c + a^2b^2 - 4b^3 - 27c^2\]
\subsubsection{根}
\bma
x =& -\frac{b}{3a}+e^{\frac{i2\pi k}{3}}\sqrt[3]{\frac{bc}{6a^2}-\frac{b^3}{27a^3}-\frac{d}{2a}+\sqrt{\left(\frac{bc}{6a^2}-\frac{b^3}{27a^3}-\frac{d}{2a}\right)^2+\left(\frac{c}{3a}-\frac{b^2}{9a^2}\right)^3}}\\
&+ e^{-\frac{i2\pi k}{3}}\sqrt[3]{\frac{bc}{6a^2}-\frac{b^3}{27a^3}-\frac{d}{2a}},\quad k=0,1,2
\eam
\subsubsection{圖形特徵}
\begin{itemize}
\item 頂點、二階旋轉對稱點、反曲點:$(h,k)$。
\item 開口:$a>0$則$x>h$時向上、$x<h$時向下;$a<0$則$x<h$時向上、$x>h$時向下。
\item 極值:$ap<0$若且惟若存在二個極值點,$p=0$若且惟若存在一個極值點(圖形單調遞增或減),$ap>0$若且惟若不存在極值點(圖形嚴格遞增或減)。
\item 最值點:不存在。
\item 發散:$a$為正,$x\to\infty$向上發散,$x\to -\infty$向下發散;$a$為負,$x\to\infty$向下發散,$x\to -\infty$向上發散。$|a|$愈大,發散愈快。
\item 頂點處斜率:$p$
\end{itemize}
\sssc{根數}
\begin{itemize}
\item $\Delta>0$:三相異實根。
\item $\Delta=0$:一實根且為三重根或二相異實根且其一為二重根。
\item $\Delta<0$:一實根,二相異共軛虛根。
\end{itemize}
\sssc{根與係數關係}
令根$\alpha$、$\beta$、$\gamma$
\[\alpha+\beta+\gamma=-\frac{b}{a}\]
\[\alpha\beta+\beta\gamma+\gamma\alpha=\frac{c}{a}\]
\[\alpha\beta\gamma=-\frac{d}{a}\]
\sssc{卡丹諾公式(Cardino's formula)}
$x^3=px+q$的判別式為:
\[\Delta=4p^3-27q^2\]
根為:
\bma
x&=e^{\frac{i2\pi k}{3}}\sqrt[3]{\frac{q}{2}+\sqrt{\frac{q^2}{4}-\frac{p^3}{27}}}+e^{-\frac{i2\pi k}{3}}\sqrt[3]{\frac{q}{2}-\sqrt{\frac{q^2}{4}-\frac{p^3}{27}}},\quad k=0,1,2\\
&=e^{\frac{i2\pi k}{3}}\sqrt[3]{\frac{q}{2}+\sqrt{\frac{-\Delta}{108}}}+e^{-\frac{i2\pi k}{3}}\sqrt[3]{\frac{q}{2}-\sqrt{\frac{-\Delta}{108}}},\quad k=0,1,2
\eam
\subsection{單變數實多項式函數的導函數}
\[\frac{\mathrm{d}}{\mathrm{d}x}x^n = n \cdot x^{n-1}\]
\begin{proof}
\bma
\frac{\mathrm{d}}{\mathrm{d}x}x^n &= \lim_{\Delta x \to 0} \frac{(x + \Delta x)^n - x^n}{\Delta x}\\
&= \lim_{\Delta x \to 0}\frac{\qty(\sum_{k=0}^{n} \binom{n}{k} x^{n-k} (\Delta x)^k) - x^n}{\Delta x}\\
&= \lim_{\Delta x \to 0}\frac{\sum_{k=1}^{n} \binom{n}{k} x^{n-k} (\Delta x)^k}{\Delta x}\\
&= \lim_{\Delta x \to 0}\sum_{k=1}^{n} \binom{n}{k} x^{n-k} (\Delta x)^{k-1}\\
&= n x^{n-1}
\eam
\end{proof}
\subsection{多項式插值法(Polynomial Interpolation)}
給定:
\[A=\{a_k\mid k\in\mathbb{N}_0\land k\leq n\}\subseteq\mathbb{R}\text{\ s.t.\ }\forall i\neq j\colon a_i\neq a_j,\]
\[B=\{b_k\mid k\in\mathbb{N}_0\land k\leq n\}\subseteq\mathbb{R},\]
多項式插值法指找​​到一個次數最多為 $n$ 的多項式 $p(x)$ 使得:
\[\forall k\in\mathbb{N}_0\land k\leq n\colon p(a_k)=b_k\]
的方法。
\subsubsection{牛頓插值法(Newton's Polynomial)}
定義差商(divided difference):
\[\begin{cases}
f[a_i]=b_i,\quad& i\in\mathbb{N}_0\land i\leq n,\\
f[a_i,a_{i+1}]=\frac{f[a_{i+1}]-f[a_i]}{a_{i+1}-a_i},\quad& i\in\mathbb{N}_0\land i\leq n-1,\\
f[(a_i)_{i=j}^k]=\frac{f[(a_i)_{i=i+1}^k]-f[(a_i)_{i=j}^{k-1}]}{a_k-a_j},\quad j<k\leq n\land j,k\in\mathbb{N}_0.
\end{cases}\]
則:
\[p(x)=\sum_{k=0}^nf[(a_i)_{i=0}^k]\prod_{i=0}^{k-1}(x-a_i).\]
\subsubsection{拉格朗日插值法(Lagrange Polynomial)}
\textbf{標準形式}:
定義基函數(base function):
\[\ell_k(x)=\prod_{a\in A\setminus\{a_k\}}\frac{x-a}{a_k-a}\]
則:
\[p(x)=\sum_{k=0}^nb_k\ell_k(x).\]
\textbf{重心形式(Barycentric form)}:
定意權重(weight):
\[w_k=\prod_{a_i\in A\setminus\{a_k\}}\frac{1}{a_k-a_i}\]
則:
\[p(x)=\begin{cases}
\frac{\sum_{k=0}^n\frac{b_kw_k}{x-a_k}}{\sum_{k=0}^n\frac{w_k}{x-a_k}},\quad& x\notin A.\\
b_k,\quad x=a_k\in A.
\end{cases}\]
\end{document}