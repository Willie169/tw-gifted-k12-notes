\documentclass[a4paper,12pt]{article}
\setcounter{secnumdepth}{5}
\setcounter{tocdepth}{3}
\newcounter{ZhRenew}
\setcounter{ZhRenew}{1}
\newcounter{SectionLanguage}
\setcounter{SectionLanguage}{1}
\input{/usr/share/latex-toolkit/template.tex}
\begin{document}
\title{複數與複數平面}
\author{沈威宇}
\date{\temtoday}
\titletocdoc
\sct{複數(Complex number)與複數平面(Complex plane)}
\ssc{複數(Complex number)}
\bit
\item \tb{複數(Complex number)}:可表示成$z=a+bi$得數,其中$a,b\in\mathbb{R}$。
\item \tb{實部(Real part)}:$z=a+bi$的實部$\Re(z)=a$。
\item \tb{虛部(Imaginary part)}:$z=a+bi$的虛部$\Im(z)=b$。
\item \tb{虛數(Imaginary number)}:虛部不為零的複數。
\item \tb{純虛數(Pure imaginary number)}:實部為零的虛數。
\eit
\ssc{共軛複數(Conjugate complex number)}
$z$的共軛複數$\ol{z}$,稱 z bar,為其實部加上其虛部的負一倍,即絕對值不變輻角乘上負號。
\sssc{複數除法}
\[\frac{1}{z}=\frac{\bar{z}}{|z|^2},z\neq 0\]
\sssc{共軛複數的性質}
\[\ol{z_1}+\ol{z_2}=\ol{z_1+z_2}.\]
\[\ol{z_1}-\ol{z_2}=\ol{z_1-z_2}.\]
\[\ol{z_1\cdot z_2}=\ol{z_1}\cdot\ol{z_2}.\]
\[\Re(\ol{z_1}\cdot z_2)=\Re(z_1\cdot\ol{z_2}).\]
\[\frac{\ol{z_1}}{\ol{z_2}}=\ol{\qty(\frac{z_1}{z_2})},\quad z_2\neq 0.\]
\[\ol{z^n}=\qty(\ol{z})^n,\quad zn\neq 0.\]
\[\ol{e^{i\theta}}=e^{-i\theta}.\]
\ssc{複數的絕對值/向徑/模(長)}
\[|a+bi|=\sqrt{a^2+b^2}\]
即:
\[|z|=\sqrt{z\cdot\ol{z}}\]
\ssc{負實數根號的性質}
\[\forall a>0\colon\sqrt{-a}=\sqrt{a}i\]
\[\sqrt{ab}=\begin{cases}
-\sqrt{a}\sqrt{b},\quad &a<0\land b<0\\
\sqrt{a}\sqrt{b},\quad &\tx{otherwise.}
\end{cases}\]
\ssc{歐拉公式(Euler's formula)}
\[e^{i\theta}=\cos\theta+i\sin\theta.\]
\ssc{多項式的性質}
\sssc{代數基本定理(Fundamental theorem of algebra)}
任何一個次數大於零的複係數$n$次方程式都至少有一個複數根。
\sssc{根數}
若$k$重根計作$k$個根,則複係數$n$次多項式方程恰有$n$個複數根。
\sssc{虛根成對定理}
實係數多項式方程$f(x)=0$如有虛根$x$則$\ol{x}$亦為其根。
\sssc{共軛複數的函數值}
實係數多項式函數$f$、複數$z$:
\[\ol{f(z)}=f\qty(\ol{z}).\]
\sssc{奇數次實係數多項式方程}
奇數次實係數多項式方程必有實根。
\ssc{複數平面(Complex plane)/阿爾岡平面(Argand plane)/高斯平面(Gaussian plane)}
由實軸為橫軸與虛軸為縱軸定義的二位笛卡爾座標平面,與複數域一一對應。
\subsection{輻角(Argument)}
$\arg(z)$代表$z$的輻角(Argument),$\operatorname{Arg}(z)$代表$z$的輻角主值/主輻角(Principal argument),惟少數文獻反之。
\subsubsection{輻角(Argument)}
$z=x+yi\in\mathbb {C}_{\neq 0}$,其中$x,y\in\mathbb{R}$,$z$的輻角$\arg(z)=\varphi$定義為使等式:
\[z=x+yi=\sqrt{x^2+y^2}(\cos \varphi +i\sin \varphi )\]
成立的任何實數$\varphi $。

$0$的輻角主值未定義。
\subsubsection{輻角主值/主輻角(Principal argument)}
$z=x+yi\in\mathbb {C} _{\neq 0}$,其中$x,y\in\mathbb{R}$,$z$的輻角主值$\operatorname{Arg}(z)$定義為其輻角中處於$(-\pi,\pi]$者,亦有文獻定義為其輻角中處於$[0,2\pi)$者。

$0$的輻角主值未定義。
\sssc{複數極式}
令複數$z$有輻角$\theta$,則其極式為:
\[z=|z|(\cos\theta+i\sin\theta).\]
\ssc{複數運算的幾何意義}
\sssc{四則運算}
\bit
\item 複數相加減,實部與虛部分別相加減,相當於平面向量相加減。
\item 複數相乘,模長相乘、輻角相加。
\item 複數相除,模長相除、輻角相減。
\eit
\sssc{隸美弗公式(De Moivre's formula)}
\[(r(\cos\theta+i\sin\theta))^n=r^n(\cos(n\theta)+i\sin(n\theta)),\quad r\neq 0\land n\in\mathbb{Z}.\]
\sssc{雙曲線函數隸美弗公式}
\[(r(\cosh\theta+i\sinh\theta))^n=r^n(\cosh(n\theta)+i\sinh(n\theta)),\quad r\neq 0\land n\in\mathbb{Z}.\]
\sssc{點積}
複數平面上$P(z_1)$與$Q(z_2)$的點積為$\Re(z_1\cdot\ol{z_2})$。
\ssc{複數的冪及其在複數平面上的幾何意義}
\sssc{1的正整數分之一次冪及其在複數平面上的幾何意義}
令$n\in\mathbb{N}$,$\omega=\cos\frac{2\pi}{n}+i\sin\frac{2\pi}{n}$,則$1^{\frac{1}{n}}$,即$x^n=1$的$n$個根,為$\omega$的0到$n-1$次方,即:
\[e^{\frac{2\pi k}{n}}=\cos\frac{2\pi k}{n}+i\sin\frac{2\pi k}{n},\quad k\in\mathbb{N}_0<n.\]
彼等根在複數平面上為以原點為圓心的單位圓的內接正$n$邊形的$n$個頂點。
\sssc{非零複數的正整數分之一次冪及其在複數平面上的幾何意義}
令$z\in\mathbb{C}_{\neq 0}\land n\in\mathbb{N}$,$\omega=\cos\frac{2\pi}{n}+i\sin\frac{2\pi}{n}$,則$z^{\frac{1}{n}}$,即$x^n=z$的$n$個根為$\sqrt[n]{|z|}$乘以$\omega$的0到$n-1$次方,即:
\[\sqrt[n]{|z|}e^{\frac{\Arg(z)+2\pi k}{n}}=\sqrt[n]{|z|}\qty(\cos\frac{\Arg(z)+2\pi k}{n}+i\sin\frac{\Arg(z)+2\pi k}{n}),\quad k\in\mathbb{N}_0<n.\]
彼等根在複數平面上為以原點為圓心、半徑$\sqrt[n]{|z|}$的圓的內接正$n$邊形的$n$個頂點。
\subsubsection{非零複數的複數次冪及其在複數平面上的幾何意義}
令$z\in\mathbb{C}_{\neq 0}\land w\in\mathbb{C}$,則$z^w$為:
\[\begin{aligned}
z^w=&e^{w\ln(z)}\\
=&\left(|z|e^{i\arg(z)}\right)^w\\
=&|z|^{\Re(w)}|z|^{\Im(w)i}e^{i\Re(w)\arg(z)}e^{-\Im(w)\arg(z)}\\
=&|z|^{\Re(w)}e^{-\Im(w)\arg(z)+i(\Re(w)\arg(z)+\Im(w)\ln|z|)}\\
=&|z|^{\Re(w)}e^{-\Im(w)\arg(z)}\left(\cos(\Re(w)\arg(z)+\Im(w)\ln|z|)+i\sin(\Re(w)\arg(z)+\Im(w)\ln|z|)\right),
\end{aligned}\]
其中:
\[\arg(z)=\Arg(z)+2\pi k,\quad k\in\mathbb{Z}.\]
彼等根在複數平面上為螺旋
\[|z|^{\Re(w)}e^{-\Im(w)\theta}\left(\cos(\Re(w)\theta+\Im(w)\ln|z|)+i\sin(\Re(w)\theta+\Im(w)\ln|z|)\right)\]
上等輻角差的點。
\paragraph*{模長:}
\[\abs{z^w}=|z|^{\Re(w)}e^{-\Im(w)\arg(z)}.\]
\begin{itemize}
\item 當$\Im(w)>0$:模長對$k$指數衰減,即隨$k$增大趨近於零、隨$k$減小發散至無限大。
\item 當$\Im(w)<0$:模長對$k$指數增長,即隨$k$增大發散至無限大、隨$k$減小趨近於零。
\item 當$\Im(w)=0$:模長始終為$|z|^{\Re(w)}$,即在複數平面上根在以原點為圓心、半徑$|z|^{\Re(w)}$的圓上。
\end{itemize}
\paragraph*{輻角:}
\[\arg\left(z^w\right)=\Re(w)\arg(z)+\Im(w)\ln|z|.\]
\begin{itemize}
\item 當$\Re(w)>0$:輻角對$k$線性增長,即隨$k$增大逆時針旋轉、隨$k$減小順時針旋轉。
\item 當$\Re(w)<0$:輻角對$k$線性衰減,即隨$k$增大順時針旋轉、隨$k$減小逆時針旋轉。
\item 當$\Re(w)\in\mathbb{Z}$:輻角主值始終為$\left(\Re(w)\Arg(z)+\Im(w)\ln|z|\right)\mod (2\pi)$,即在複數平面上根在輻角$\Re(w)\Arg(z)+\Im(w)\ln|z|$的射線上。
\end{itemize}
\paragraph*{螺旋方向:}
\begin{itemize}
\item 當$\Re(w)\Im(w)>0$:在複數平面上根在以原點為中心點的順時針發散的螺旋上。
\item 當$\Re(w)\Im(w)<0$:在複數平面上根在以原點為中心點的逆時針發散的螺旋上。
\item 當$\Im(w)=0$:在複數平面上根在以原點為圓心的圓上。
\end{itemize}
\paragraph*{根的個數:}
\begin{itemize}
\item 當$\Im(w)=0\land\Re(w)\in\mathbb{Q}$:令$\abs{\Re(w)}=\frac{m}{n}$其中$n\in\mathbb{N},\gcd(m,n)=1$,則有$n$個根,且彼等根在複數平面上為以原點為圓心、半徑$|z|^{\Re(w)}$的圓的內接正$n$邊形的$n$個頂點,且其中一個頂點在輻角$\Re(w)\Arg(z)$的射線上。
\item 其他情況:有無限多個根。
\end{itemize}
\end{document}