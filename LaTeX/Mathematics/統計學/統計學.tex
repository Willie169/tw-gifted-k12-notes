\documentclass[a4paper,12pt]{report}
\setcounter{secnumdepth}{5}
\setcounter{tocdepth}{3}
\newcounter{ZhRenew}
\setcounter{ZhRenew}{1}
\newcounter{SectionLanguage}
\setcounter{SectionLanguage}{1}
\input{/usr/share/latex-toolkit/template.tex}
\begin{document}
\title{統計學}
\author{沈威宇}
\date{\temtoday}
\titletocdoc
\chapter{統計學(Statistics)}
\section{一維數據分析}
今有一由小到大排列的實數序列\(\mathbf{X}=(x_1,x_2,\ldots,x_n)\)。
\subsection{眾數(Mode, Mo)}
$\text{Mode}(\mb{X})$,指$\mathbf{X}$中出現次數最多者。
\subsection{中位數(Median, Me)}
\[ \text{Median}(\mb{X})=\begin{cases}
x_{\frac{n+1}{2}},\quad & n\text{\ is odd.}\\
\frac{x_{\frac{n}{2}}+x_{\frac{n}{2}+1}}{2},\quad & n\text{\ is even.}
\end{cases} \]
\subsection{(算術)平均數((Arithmetic) mean)}
\[\mu=\frac{\sum_{i=1}^n x_i}{n}\]
\subsection{加權平均數(Weighted mean)}
令\((x_1,x_2,\ldots,x_n)\)對應的權數/權重(weight)為\((w_1,w_2,\ldots,w_n)\),則加權平均數為:
\[\frac{\sum_{i=1}^n x_iw_i}{\sum_{i=1}^n w_i}\]
\subsection{幾何平均數}
\[\sqrt[n]{\prod_{i=1}^n x_i}\]
\subsection{百分位數(Percentile, Percentile score)}
令:第$k$百分位數為$P_k$,$m=n\frac{k}{100}$,$i=\lfloor m\rfloor$,$j=i+1$,$g=m-i$,$h=\frac{k}{100}\qty(n-\alpha -\beta +1)+\alpha$,$r=(n-1)\frac{k}{100}$,$s=\lfloor r\rfloor+1$,$t=s+1$。\\
令:
\[\begin{aligned}
& \text{RoundHalfToEven}(x) = \left\{
  \begin{array}{ll}
  \lfloor x \rfloor, & \text{if\ } x - \lfloor x \rfloor < 0.5 \\
  \lceil x \rceil, & \text{if\ } x - \lfloor x \rfloor > 0.5 \\
  2\left\lfloor \frac{x}{2} \right\rfloor, & \text{if } x - \lfloor x \rfloor = 0.5 \text{\ and\ } \lfloor x \rfloor \text{\ is even} \\
  2\left\lfloor \frac{x}{2} \right\rfloor + 1, & \text{if } x - \lfloor x \rfloor = 0.5 \text{\ and\ } \lfloor x \rfloor \text{\ is odd}
  \end{array}
\right.\\
& x_h = x_{\left\lfloor h \right\rfloor} + (h-\lfloor h\rfloor)(x_{\left\lceil h \right\rceil}-x_{\left\lfloor h \right\rfloor}).
\end{aligned}\]
各種百分位數定義主要分為兩類:
\begin{itemize}
\item 包含性定義(Inclusive definition):較常用。至少有$k$\%的項$\leq P_k$,且至少有$(100-k)\%$的項$\geq P_k$。
\item 排他性定義(Exclusive definition):較少用。至少有$k\%$的項$<P_k$,且至少有$(100-k)\%$的項$>P_k$。
\end{itemize}
各種百分位數定義:
\begin{itemize}
\item inverted\_ cdf (method 1 of H \& F):
\[ P_k=\begin{cases}
x_j,g>0\\
x_i,g=0
\end{cases}\]
\item averaged\_ inverted\_ cdf (method 2 of H \& F): 離散定義中最常用。
\[ P_k=\begin{cases}
x_j,g>0\\
\frac{x_i+x_j}{2},g=0
\end{cases}\]
\item closest\_ observation (method 3 of H \& F):
\[ P_k=x_{\text{RoundHalfToEven}(m)} \]
\item interpolated\_ inverted\_ cdf (method 4 of H \& F):
\bma
\alpha &= 0\\
\beta &= 1\\
P_k &= x_h
\eam
\item hazen (method 5 of H \& F):
\bma
\alpha &= \frac{1}{2}\\
\beta &= \frac{1}{2}\\
P_k &= x_h
\eam
\item weibull (method 6 of H \& F): Excel PERCENTILE.EXC 使用其乘以\%為值。
\bma
\alpha &= 0\\
\beta &= 0\\
P_k &= x_h
\eam
\item linear (method 7 of H \& F): Excel PERCENTILE.INC 使用其乘以\%為值。連續定義中最常用。
\bma
\alpha &= 1\\
\beta &= 1\\
P_k &= x_h
\eam
\item median\_ unbiased (method 8 of H \& F):
\bma
\alpha &= \frac{1}{3}\\
\beta &= \frac{1}{3}\\
P_k &= x_h
\eam
\item normal\_ unbiased (method 9 of H \& F):
\bma
\alpha &= \frac{3}{8}\\
\beta &= \frac{3}{8}\\
P_k &= x_h
\eam
\item lower (NumPy old method):
\[P_k=x_s\]
\item higher (NumPy old method):
\[P_k=x_t\]
\item nearest (NumPy old method):
\[P_k=x_{\text{RoundHalfToEven}(r)+1}\]
\item midpoint (NumPy old method):
\[P_k=\frac{x_s+x_t}{2}\]
\end{itemize}
\subsection{四分位數(Quantile)}
與百分位數同有該等各種定義,僅將其中之100均改為4、第$k$四分位數稱$Q_k$、PERCENTILE.INC改為QUANTILE.INC、PERCENTILE.EXC改為QUANTILE.EXC,並另有下列其他定義方法。\\
\begin{itemize}
\item 定義一:先取中位數為$Q_2$,將序列以中位數為界分為兩半,若$n$為奇數則中位數不包含在兩半,分別取兩半之中位數為$Q_1$、$Q_3$。
\item 定義二:先取中位數為$Q_2$,將序列以中位數為界分為兩半,若$n$為奇數則中位數包含在兩半,分別取兩半之中位數為$Q_1$、$Q_3$。
\item 定義三:先取中位數為$Q_2$,若$n$為偶數則將序列以中位數為界分為兩半,分別取兩半之中位數為$Q_1$、$Q_3$;若$n$除以4的商為$q$且餘數為1,則$Q_1=0.25x_q+0.75x_{q+1}$;$Q_3=0.75x_{3q+1}+0.25x_{3q+2}$;若$n$除以4的商為$q$且餘數為3,則$Q_1=0.75x_{q+1}+0.25x_{q+2}$;$Q_3=0.25x_{3q+2}+0.75x_{3q+3}$。
\end{itemize}
\ssc{百分位排名/等級(Percentile rank)}
令百分位等級PR,累積次數CF為小於等於感興趣值的項數,次數F為於等於感興趣值的項數,CF'為小於感興趣值的項數。
\bit
\item 定義一:
\[\tx{PR}=100\frac{\tx{CF}-0.5\tx{F}}{n}=100\frac{\tx{CF'}+0.5\tx{F}}{n}\]
\item 定義二(Excel PERCENTRANK.INC定義),最常用:
\[\tx{PR} = 100\frac{CF'}{n-1}\]
\item 定義三(Excel PERCENTRANK.EXC定義):
\[\tx{PR} = 100\frac{CF'+1}{n+1}\]
\eit
\ssc{全距}
\[R=\max(\mathbf{X})-\min(\mathbf{X})\]
\ssc{四分位距}
\[Q_3-Q_1\]
\ssc{母體變異數(Population variance)和母體標準差(Population standard deviation)}
稱$x_i-\mu$為離均差,$i=1,2,\ldots,n$。母體變異數$\sigma^2$或$\operatorname{Var}(X)$,母體標準差$\sigma$:
\bma
\sigma^2 &= \frac{\sum_{i=1}^n (x_i-\mu)^2}{n}\\
&= \frac{\sum_{i=1}^n x_i^{\pht{i}2}}{n}-\mu^2\\
\sigma &= \sqrt{\sigma^2}
\eam
\ssc{樣本變異數(Sample variance)和樣本標準差(Sample standard deviation)}
稱$x_i-\mu$為離均差,$i=1,2,\ldots,n$。樣本變異數$s^2$,樣本標準差$s$:
\bma
s^2 &= \frac{\sum_{i=1}^n (x_i-\mu)^2}{n-1}\\
&= \frac{\sum_{i=1}^n x_i^{\pht{i}2}-n\mu^2}{n-1}\\
s &= \sqrt{s^2}
\eam
\ssc{仿射變換}
$\mathbf{X}$的仿射變換$\mathbf{Y}=(y_i\mid y_i=ax_i+b,i=1,2,\ldots,n)$,記作$\mathbf{Y}=a\mathbf{X}+b$。\\
性質:
\[\mu_\mathbf{Y}=a\mu_\mathbf{X}+b\]
\[\sigma_\mathbf{Y}=|a|\sigma_\mathbf{X}\]
\[s_\mathbf{Y}=|a|s_\mathbf{X}\]
\[\text{Mode}(\mb{Y})=a\text{Mode}(\mb{X})+b.\]
\[\text{Median}(\mb{Y})=a\text{Median}(\mb{X})+b.\]
\ssc{標準分數(Standard score)/Z分數(z-score)}
$\mathbf{Z}$,定義為:
\[\mathbf{Z}=\left\{z_i\middle| z_i=\frac{x_i-\mu}{\sigma},i=1,2,\ldots,n\right\}.\]
將數據轉換為標準分數的過程稱標準化(standardizing or normalizing)。

性質:
\[\mu_\mathbf{Z}=0,\quad\sigma_\mathbf{Z}=1\]
\[\sum_{i=1}^n z_i^{\pht{i}2}=n\]
\sct{二維數據分析}
今有由小到大排列的實數序列\(\mathbf{X}=(x_1,x_2,\ldots,x_n)\)與\(\mathbf{Y}=(y_1,y_2,\ldots,y_n)\),標準差分別為$\sigma_\mathbf{X},\sigma_\mathbf{Y}$,算術平均數分別為$\mu_\mathbf{X},\mu_\mathbf{Y}$,標準分數分別為\(\mathbf{X}'=\qty(x_i' \middle | x_i'=\frac{x_i-\mu_\mathbf{X}}{\sigma_\mathbf{X}},i=1,2,\ldots,n)\)與\(\mathbf{Y}'=\qty(y_i' \middle | y_i'=\frac{y_i-\mu_\mathbf{Y}}{\sigma_\mathbf{Y}},i=1,2,\ldots,n)\)。
\ssc{散布圖}
將數據點每個$(x_i,y_i),i=1,2,\ldots,n$描繪在$xy$平面直角座標平面。
\ssc{皮爾森積動差相關係數(Pearson product-moment correlation coefficient, PPMCC, PCC)/相關係數(Correlation coefficient)}
$\mathbf{X}$與$\mathbf{Y}$的相關係數記作$r_{\mathbf{X}\mathbf{Y}}$。定義:
\bma
r_{\mathbf{X}\mathbf{Y}} &= \frac{\sum_{i=1}^n x_i'y_i'}{n}\quad\tx{(標準化積和除以項數)}\\
S_{\mathbf{X}\mathbf{X}} &= \sum_{i=1}^n (x_i-\mu_\mathbf{X})^2=\sum_{i=1}^n x_i^2 - n\mu_\mathbf{X}^{\pht{\mathbf{X}}2}=n\sigma_\mathbf{X}^{\pht{\mathbf{X}}2}\\
S_{\mathbf{Y}\mathbf{Y}} &= \sum_{i=1}^n (y_i-\mu_\mathbf{Y})^2=\sum_{i=1}^n y_i^2 - n\mu_\mathbf{Y}^{\pht{\mathbf{Y}}2}=n\sigma_\mathbf{Y}^{\pht{\mathbf{Y}}2}\\
S_{\mathbf{X}\mathbf{Y}} &= \sum_{i=1}^n (x_i-\mu_\mathbf{X})(y_i-\mu_\mathbf{Y})=\sum_{i=1}^n x_iy_i - n\mu_\mathbf{X}\mu_\mathbf{Y}\\
r_{\mathbf{X}\mathbf{Y}} &= \frac{S_{\mathbf{X}\mathbf{Y}}}{\sqrt{S_{\mathbf{X}\mathbf{X}}S_{\mathbf{Y}\mathbf{Y}}}}\quad\tx{(離均差積和除以根號離均差平方和積)}\\
\eam
性質:
\[-1\leq r\leq 1,\quad 0\leq r^2\leq 1\]
\[r_{\mathbf{X}\mathbf{Y}}=r_{\mathbf{Y}\mathbf{X}}\]
相關程度:
\begin{itemize}
\item $r=1$稱完全正相關;$r=-1$稱完全負相關。
\item $r>0$稱正相關;$r<0$稱負相關;$r=0$稱無無相關。
\end{itemize}
仿射變換:\\
令$\mathbf{X}'=a\mathbf{X}+b$、$\mathbf{Y}'=a\mathbf{Y}+b$。
\[r_{\mathbf{X}'\mathbf{Y}'}=\frac{ac}{\qty|ac|}r_{\mathbf{X}\mathbf{Y}}\]
\ssc{判定係數(Coefficient of determination)}
指皮爾森積動差相關係數的平方。
\ssc{(線性)迴歸直線/最適直線}
令平方和:
\[D=\sum_{i=1}^n \qty(y_i-(mx_i+k))^2\]
解出使$D$最小(即$D$為最小平方和)的$m,k$即得$\mathbf{L}$(即最小平方法)。\\
\(\mathbf{X}'\)與\(\mathbf{Y}'\)的最適直線$\mathbf{L}:\,mx+k$為:
\[y'=r_{\mathbf{X}'\mathbf{Y}'}x'\]
\(\mathbf{X}\)與\(\mathbf{Y}\)的最適直線為:
\[y-\mu_\mathbf{Y}=m(x-\mu_\mathbf{X})\]
其中:
\[m=r_{\mathbf{X}\mathbf{Y}}\cdot\frac{\sigma_\mathbf{Y}}{\sigma_\mathbf{X}}=\frac{S_{\mathbf{X}\mathbf{Y}}}{S_{\mathbf{X}\mathbf{X}}}\]
\sct{多維資料分析}
\ssc{迴歸直線}
設有 \( n \) 個樣本,每個樣本有 \( m \) 個特徵。\\
令矩陣\(\mathbf{X}\) 是 \( n \times (m+1) \) 的矩陣,第一 column 是全為1 的 column (對應截距項),其餘 column 是特徵\( x_1, x_2,\ldots,x_m \)。\\
令 \(\mathbf{y}\) 是 \( n \times 1 \) 的 column 向量,表示目標變數。\\
\[
\mathbf{X} = \begin{pmatrix}
1 & x_{11} & x_{12} & \cdots & x_{1m} \\
1 & x_{21} & x_{22} & \cdots & x_{2m} \\
\vdots & \vdots & \vdots & \ddots & \vdots \\
1 & x_{n1} & x_{n2} & \cdots & x_{nm}
\end{pmatrix}
\]
\[
\mathbf{y} = \begin{pmatrix}
y_1 \\
y_2 \\
\vdots \\
y_n
\end{pmatrix}
\]
迴歸係數 \(\mathbf{a}\) 可以用以下公式計算:
\[
\mathbf{a} = (\mathbf{X}^T \mathbf{X})^{-1} \mathbf{X}^T \mathbf{y}
\]
得到迴歸方程式:
\[
y=(1,x_1, x_2,\ldots,x_m)\mathbf{a}
\]
\begin{proof}\mbox{}\\
最小平方法的目標是找到一組係數 \(\mathbf{a}\),使得實際值 \(\mathbf{y}\) 與預測值 \(\mathbf{X}\mathbf{a}\) 之間的平方差和最小,即最小化以下目標函數:
\[
J(\mathbf{a}) = \sum_{i=1}^n (y_i - \mathbf{X}_i \mathbf{a})^2
\]
其中,\(\mathbf{X}_i\) 是 \(\mathbf{X}\) 的第 \(i\) row。
寫成矩陣形式:
\[
J(\mathbf{a}) = (\mathbf{y} - \mathbf{X}\mathbf{a})^T (\mathbf{y} - \mathbf{X}\mathbf{a})
\]
展開:
\[
J(\mathbf{a}) = \mathbf{y}^T\mathbf{y} - \mathbf{y}^T\mathbf{X}\mathbf{a} - \mathbf{a}^T\mathbf{X}^T\mathbf{y} + \mathbf{a}^T\mathbf{X}^T\mathbf{X}\mathbf{a}
\]
因純量的轉置為其自身,所以:
\[\mathbf{y}^T\mathbf{X}\mathbf{a}=\mathbf{a}^T\mathbf{X}^T\mathbf{y}\]
即:
\[
J(\mathbf{a}) = \mathbf{y}^T\mathbf{y} - 2\mathbf{y}^T\mathbf{X}\mathbf{a} + \mathbf{a}^T\mathbf{X}^T\mathbf{X}\mathbf{a}
\]
要最小化 \(J(\mathbf{a})\),我們對 \(\mathbf{a}\) 求導數並令其為零:
\[
\frac{\partial J(\mathbf{a})}{\partial \mathbf{a}} = -2\mathbf{y}^T\mathbf{X} + \mathbf{X}^T \mathbf{X}\mathbf{a} = 0
\]
整理後得到:
\[
\mathbf{X}^T \mathbf{X} \mathbf{a} = \mathbf{X}^T \mathbf{y}
\]
假設 \(\mathbf{X}^T \mathbf{X}\) 是可逆的,我們可以兩邊同時乘以 \((\mathbf{X}^T \mathbf{X})^{-1}\):
\[
\mathbf{a} = (\mathbf{X}^T \mathbf{X})^{-1} \mathbf{X}^T \mathbf{y}
\]
\end{proof}
\sct{Statistical hypothesis test (統計假說檢定)}
A statistical hypothesis test is a method of statistical inference used to decide whether the data sufficiently supports a particular hypothesis.
\ssc{Terms}
\begin{itemize}
\item Statistical hypothesis (統計假說): A statement about the parameters describing a population (not a sample).
\item Test statistic (統計量) $T$: A value calculated from a sample without any unknown parameters, often to summarize the sample for comparison purposes.
\item Simple hypothesis (簡單假說): Any hypothesis which specifies the population distribution completely.
\item Null hypothesis (虛無假說) $H_0$: The claim that the effect being studied does not exist.
\item Positive data (陽性數據): Data that enable the investigator to reject a null hypothesis.
\item Alternative hypothesis (對立假說) $H_1$: One of the proposed propositions in the hypothesis test.
\item Region of rejection (拒絕域) or Critical region: The set of values of the test statistic for which the null hypothesis is rejected.
\item Power of test (檢定力) $1-\beta$: In the case of a simple hypothesis test with two hypotheses, the power of the test is the probability that the test correctly rejects the null hypothesis $H_0$ when the alternative hypothesis $H_1$ is true.
\item Size or False positive rate (偽陽性率); For simple hypotheses, this is the test's probability of incorrectly rejecting the null hypothesis.
\item p-value (p值) $p$: The probability of obtaining a result at least as extreme, given that the null hypothesis is true.
\item Significance level (顯著水準) $\alpha$: The probability of the study rejecting the null hypothesis, given that the null hypothesis is true. The result is statistically significant, by the standards of the study, when $p<\alpha$.
\item Most powerful test: For a given size or significance level, the test with the greatest power (probability of rejection) for a given value of the parameter(s) being tested, contained in the alternative hypothesis.
\item Uniformly most powerful test (UMP): The hypothesis test which has the greatest power $1-\beta$ among all possible tests of a given size. 
\end{itemize}
\ssc{Steps}
The typical steps involved in performing a frequentist hypothesis test in practice are:
\begin{enumerate}
\item Define a hypothesis (claim which is testable using data).
\item Select a relevant statistical test with associated test statistic $T$.
\item Derive the distribution of the test statistic under the null hypothesis from the assumptions.
\item Select a significance level $\alpha$.
\item Compute from the observations the observed value $t_{obs}$ of the test statistic $T$.
\item Decide to either reject the null hypothesis in favor of the alternative hypothesis or not reject it. Reject the null hypothesis $H_0$ if the observed value $t_{obs}$ is in the critical region, and not reject the null hypothesis otherwise.
\end{enumerate}
\sct{參考文獻}
\begin{itemize}
\item R. J. Hyndman and Y. Fan, “Sample quantiles in statistical packages,” The American Statistician, 50(4), pp. 361-365, 1996.
\item Numpy. numpy.percentile. \href{https://numpy.org/doc/stable/reference/generated/numpy.percentile.html.}{https://numpy.org/doc/stable/reference/generated/numpy.percentile.html.}
\item Wikipedia. \href{https://wikipedia.org}{https://wikipedia.org}.
\end{itemize}
\end{document}